\capitulo{1}{Introducción}

El análisis de sentimientos intenta descubrir la actitud de un usuario con respecto a algún tema.
Esta puede ser una reseña, un estado afectivo o emocional. Este de análisis se denomina minería de opinión.
El procesamiento de lenguaje natural busca mecanismos para la comunicación entre personas y máquinas por medio del
lenguaje natural. Se pueden realizar distintos tipos de análisis: sintáctico, semántico, prágmatico y planificación de frases.
En principio el objetivo del análisis de sentimientos es clasificar la polaridad del texto. (positiva, negativa o neutra).
Existen muchas formas de llevar a cabo este análisis (localización de palabras, afinidad léxica, métodos estadísticos o tecnicas conceptuales)
En este trabajo vamos a evaluar la utilización del "Prompt Engineering" asociado a un modelo grande de lenguaje (LLM) para ese análisis de sentimiento.
Evaluaremos su viabilidad y sus resultados. Usaremos dos medidas para evaluar el modelo, por un lado la precisión (número de elementos entre el número total) 
y la exhaustividad (recall) que mide la cantidad de elementos que ha sido capaz de identificar (se calcula dividiendo el número de aciertos entre el número 
total que hemos clasificado de esa forma). Se considera un modelo equiparable a un humano si consigue un 70 por ciento de aciertos.

Descripción del contenido del trabajo y del estructura de la memoria y del resto de materiales entregados.
