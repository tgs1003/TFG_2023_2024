\capitulo{6}{Trabajos relacionados}

\section{BERT}

BERT~\cite{devlin2019bert}, que significa ``\emph{Bidirectional Encoder Representations from Transformers}''
(Representaciones de Codificador Bidireccional a partir de Transformadores),
es un modelo de lenguaje desarrollado por Google en 2018.\\
BERT es parte de la familia de modelos de lenguaje basados en transformers
y es ampliamente conocido por su capacidad para comprender el contexto
de las palabras en un texto, lo que lo convierte en una herramienta
poderosa en el análisis de sentimiento y muchas otras tareas de procesamiento
del lenguaje natural (NLP).\\
Lo que hace que BERT sea particularmente efectivo en el análisis de sentimiento
es su capacidad para entender la semántica y el significado contextual de las palabras en un texto.
A diferencia de los modelos de lenguaje unidireccionales anteriores,
BERT es bidireccional, lo que significa que procesa el texto de izquierda a derecha y
de derecha a izquierda en dos pasadas separadas.\\
Esto le permite capturar el contexto de una palabra en función de las
palabras que la rodean en todas las direcciones, lo que mejora significativamente
su comprensión del lenguaje natural.\\
En el análisis de sentimiento, BERT se utiliza para evaluar el tono
o polaridad del texto, es decir, determinar si el texto tiene un sentimiento positivo,
negativo o neutral.\\ 
Para hacerlo, se entrena a BERT~\cite{BertSentimentModel1}
en grandes conjuntos de datos etiquetados con sentimientos, lo que le permite aprender
a identificar patrones y contexto asociados con emociones específicas en el lenguaje.\\
BERT ha sido un avance significativo en el campo del procesamiento
del lenguaje natural y ha mejorado considerablemente el rendimiento
en una amplia gama de aplicaciones, incluido el análisis de sentimiento,
la traducción automática, la respuesta a preguntas y muchas otras tareas de NLP.

\section{NLTK}
NLTK~\cite{NLTK1} es una biblioteca de código abierto en el lenguaje de programación 
Python que proporciona herramientas, recursos y bibliotecas para trabajar 
con el procesamiento de lenguaje natural (NLP).\\
Fue desarrollada originalmente por Steven Bird y Edward Loper en la Universidad de Pensilvania.
NLTK es ampliamente utilizado en la comunidad de NLP y es una herramienta esencial 
para investigadores, estudiantes y desarrolladores que trabajan en tareas relacionadas 
con el procesamiento del lenguaje natural.\\
NLTK proporciona una variedad de herramientas para tokenizar, segmentar, etiquetar y analizar texto.
Incluye una amplia gama de recursos lingüísticos, como corpus (conjuntos de datos de texto etiquetado), 
léxicos y otros datos que son útiles para la investigación y desarrollo en NLP.\\
Ofrece módulos y funciones para llevar a cabo tareas como análisis sintáctico, 
análisis semántico, desambiguación de sentidos, análisis de sentimiento y más.\\
NLTK se integra con otras bibliotecas y recursos, lo que facilita la conexión 
con motores de búsqueda, bases de datos y otras herramientas de procesamiento del lenguaje natural.\\
NLTK también incluye funcionalidades para el aprendizaje automático en NLP, 
lo que permite a los usuarios desarrollar modelos de lenguaje y clasificadores.\\
NLTK es muy útil para quienes deseen aprender sobre procesamiento de lenguaje natural, 
ya que proporciona una base sólida y muchas herramientas prácticas.\\
Además, es una excelente opción para prototipar y desarrollar soluciones de NLP en Python~\cite{NLTK2}.


\section{Scikit-learn}
Scikit-learn es una biblioteca de aprendizaje automático en Python que se utiliza 
ampliamente en una variedad de aplicaciones, incluido el análisis de sentimiento.\\ 
Sin embargo, en el contexto específico del análisis de sentimiento, 
scikit-learn se utiliza más como una herramienta para la construcción 
de modelos de clasificación y evaluación de rendimiento que para el 
procesamiento del lenguaje natural en sí.\\
En el análisis de sentimiento, scikit-learn~\cite{scikit-learn-sentiment1} se utiliza 
para extracción de textos, construcción de modelos (clasificación, SVM, Naive Bayes, Regresión, etc) y
evaluación de rendimiento (precisión, recall, F1-score y matriz de confusión).

En resumen, scikit-learn es una herramienta valiosa en el análisis de sentimiento, 
ya que facilita la construcción y evaluación de modelos de clasificación 
para determinar el sentimiento en textos.\\ 
Sin embargo, para tareas de procesamiento del lenguaje natural 
más complejas, como la comprensión del contexto y la semántica del texto, 
a menudo se combinan otras bibliotecas y modelos NLP, como spaCy, 
transformers (como BERT o GPT), y NLTK, con scikit-learn para lograr u
n rendimiento óptimo en el análisis de sentimiento.


\section{Sentinel}
Sentinel~\cite{Sentinel1} es una aplicación web desarrollada como TFG del grado de ingeniería 
informática de la Universidad de Burgos que realiza un análisis de sentimientos de textos 
extraídos de las redes sociales Twitter e Instagram.\\
La aplicación busca los tweets relacionados con la palabra introducida 
en el caso de Twitter, o los comentarios que le han escrito a la 
cuenta de Instagram que se ha buscado.\\
Después analiza el sentimiento que hay en ellos, los puntúa con valores entre 0 y 1 y 
se almacenan los resultados.\\
Estos resultados se muestran al usuario en gráficos 
y tablas para hacer la experiencia más visual.\\ 
Además se le ofrece la opción de calcular series temporales a partir de los resultados, 
y se da una predicción de los valores futuros.


\section{Lingmotif~\cite{lingmotif}}
Lingmotif es una aplicación multi-plataforma de sobremesa que analiza textos desde la perspectiva 
del Análisis de Sentimiento. \\Básicamente, es capaz de determinar la orientación semántica 
(si es positivo o negativo y en qué grado) de un texto o conjunto de textos, mediante \textbf{la detección 
de expresiones lingüísticas} que indican una determinada polaridad.

A diferencia de la mayoría del software existente, Lingmotif no es un sólo un clasificador, 
ya que no se limita a clasificar un texto como positivo o negativo, sino que además ofrece una
 serie de datos cuantitativos, una visualización del ``perfil de sentimiento'' del texto o textos 
 (incluyendo series temporales), y un detallado análisis cualitativo del texto en sí, en el que 
 se muestran los segmentos textuales identificados. \\Estas funcionalidades lo convierten en un herramienta 
 única, y sus aplicaciones van más allá de las que normalmente ofrece este tipo de software. 
 Lingmotif ofrece los resultados de sus análisis en archivos con formato HTML, con la versatilidad y 
 fácil manejo que esto supone.

Actualmente Lingmotif analiza textos en español e inglés. Se está trabajando en nuevas versiones 
para alemán e italiano.

\section{\href{https://www.danielsoper.com/sentimentanalysis/default.aspx}{Free Sentiment Analysis}}
Es una herramienta gratuita que permite realizar un analisis de sentimiento de cualquier 
texto escrito en Inglés. La herramienta califica el sentimiento con un número entre -100 y 100 
dependiendo del sentimiento detectado en el texto.
Sólo hay que pegar el texto en el cuadro de texto y pulsar el botón ``Analyze text''

Para su funcionamiento utiliza algoritmos de linguística computacional y minería de texto.
El modelo se ha entrenado usando el ``American National Corpus'' con lo que sólo funciona con 
textos en inglés americano y escritos después de 1990. 
No se conoce su funcionamiento interno pero por la descripción parece que ha usado un 
corpus anotado y usa las coincidencias de palabras para dar una nota al texto dependiendo 
de las palabras encontradas.

\section{\href{https://www.lexalytics.com/}{Lexalytics}}
Lexalytics es una empresa que se especializa en tecnologías de análisis de texto y 
procesamiento del lenguaje natural (PLN). 
Ofrece soluciones para extraer información valiosa de datos de texto no estructurados, 
como contenido de redes sociales, comentarios de clientes, reseñas y más. 
Los productos de Lexalytics están diseñados para ayudar a las empresas a analizar 
y comprender el sentimiento, los temas y las tendencias dentro de grandes volúmenes de información textual.
Uno de sus productos destacados es la Plataforma de Inteligencia Lexalytics, 
que incluye varias herramientas y funciones para el análisis de texto. 
Algunos de los componentes clave de su plataforma incluyen:
Salience: Este es el motor central de análisis de texto de Lexalytics, utilizado 
para el análisis de sentimientos, el reconocimiento de entidades y la extracción de temas clave de texto.
Semantria: Es la API de análisis de texto y sentimiento basada en la nube de Lexalytics, 
que brinda a los desarrolladores acceso a potentes capacidades de PLN.
Servicios de Extracción de Datos (DES): DES está diseñado para extraer automáticamente 
información e ideas de varios tipos de documentos, como contratos, correos electrónicos y 
otro contenido relacionado con negocios.
Soporte para Múltiples Idiomas: Las soluciones de Lexalytics admiten varios idiomas, 
lo que las hace adecuadas para empresas con presencia global.
Estas herramientas y servicios pueden ser valiosos para las organizaciones que 
buscan dar sentido a la gran cantidad de datos de texto no estructurados que generan o 
encuentran en el curso de sus operaciones. Se pueden aplicar en áreas como la gestión 
de la experiencia del cliente, el monitoreo de redes sociales, la investigación de mercado 
y otros campos donde comprender la información textual es crucial. 

\section{IBM Watson}
IBM Watson es una plataforma de inteligencia artificial (IA) desarrollada por IBM que proporciona servicios y herramientas avanzadas para el procesamiento del lenguaje natural, el aprendizaje automático, la analítica de datos y otras capacidades cognitivas. Se llama Watson en honor al fundador de IBM, Thomas J. Watson. La plataforma Watson se ha utilizado en una variedad de aplicaciones y sectores, desde la atención médica y la educación hasta el comercio y los servicios financieros.
Algunas de las capacidades clave de IBM Watson incluyen:
Procesamiento del Lenguaje Natural (NLP): Watson puede comprender y analizar el lenguaje humano, 
permitiendo la interpretación de preguntas y la generación de respuestas significativas.
Aprendizaje Automático y Analítica Predictiva: Watson utiliza técnicas de aprendizaje automático 
para extraer patrones a partir de grandes conjuntos de datos y realizar análisis predictivos.
Visión por Computadora: Capacidades para analizar y comprender imágenes y videos, lo que puede 
ser útil en aplicaciones como el reconocimiento facial y la interpretación de contenido visual.
Asesoramiento de Expertos: Watson ha sido utilizado en aplicaciones como juegos de preguntas y 
respuestas, demostrando su capacidad para competir y ganar contra humanos en contextos complejos.
Automatización de Procesos Empresariales: Watson puede ser utilizado para automatizar tareas y 
procesos empresariales, mejorando la eficiencia operativa.
Análisis de Datos en Tiempo Real: Watson puede analizar datos en tiempo real para proporcionar 
información instantánea y tomar decisiones basadas en datos.
IBM Watson ha sido aplicado en diversos campos, como la atención médica, donde se ha utilizado 
para ayudar en el diagnóstico de enfermedades, en el ámbito financiero para el análisis de riesgos, 
en la educación para personalizar la enseñanza, y en muchos otros sectores.
