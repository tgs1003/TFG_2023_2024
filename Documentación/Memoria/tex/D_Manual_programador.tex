\apendice{Documentación técnica de programación}

\section{Introducción}

En este apéndice se va a explicar cómo está organizado el proyecto, el
manual del programador y los requisitos necesarios para poder ejecutar el
proyecto.


\section{Estructura de directorios}

En el proyecto hay una estructura de directorios que intenta organizar el trabajo de manera lógica:
Por un lado tenemos el código y por otra la documentación.
Dentro del código podemos apreciar una carpeta llamada colab con un cuaderno de Jupyter para poder ejecutarlo en Colab.
Hay otra carpeta con el código del proyecto es sí.
La estructura de directorios quedaría así:

\begin{itemize}
	\item Código
	\begin{itemize}
		\item colab: Contiene un cuaderno de Jupyter preparado para ejecutar un ejemplo en Colab.
		\item prompt\_sentiment: Contiene el código del trabajo.
		\begin{itemize}
			\item frontend: Contiene el código de la interfaz de usuario. Está hecha con Vue, usando el plugin Vuetify.
			\item backend: Contiene el código del servicio web de la aplicación. Está hecha en Flask (python) usando Flask-restx y SQLAlchemy para el acceso a las base de datos.
			\item ngix: En este directorio está la configuración del servidor ngix que se usa en la aplicación como proxy inverso y para gestionar la configuración https.
		\end{itemize}
	\end{itemize}
	\item Documentación
	\begin{itemize}
		\item Manual de usuario: El manual de uso de la aplicación
		\item Memoria: Es la memoria del proyecto
	\end{itemize}
\end{itemize}

\section{Manual del programador}

\subsection {Instalación local}
Para la instalación local del proyecto tenemos realizar las siguientes instalaciones:

\subsubsection {Instalación de Python 3 ~\cite{lutz2001programming}}
Para ejecutar nuestro proyecto es necesario instalar Python, para ello
podemos descargar la versión aquí: \url{https://www.python.org/downloads/}
Cuando la descarga haya finalizado ejecutamos e instalamos.
También se recomienda usar conda para crear un entorno virtual para python. 
\url{https://docs.conda.io/projects/conda/en/latest/index.html}

\subsubsection {Instalación de node.js}
Para la parte de angular es necesario instalar Nodejs, por tanto accederemos 
a la página oficial, \url{https://nodejs.org/es/download/}, 
y descargaremos la versión que necesitemos y llevaremos a cabo su instalación

\subsubsection {Instalación de PostgreSql}
Para la instalación de PostgreSql tenemos que bajarnos la versión adecuada a nuestro sistema operativo e instalarla:
\url{https://www.postgresql.org/download/}

\subsection {Docker}
Para la instalación de docker sólo hay que instalar Docker-Desktop.
\subsubsection {Instalación de Docker Desktop}
\url{https://docs.docker.com/engine/install/}

\section{Compilación, instalación y ejecución del proyecto}
\subsection {Ejecución local}
El proyecto tiene 2 partes claramente diferenciadas: Frontend y Backend.
Para la ejecución del backend tenemos que abrir una consola de comandos, movernos al directorio backend 
y ejecutar:
//Instalamos las referencias
pip install -r requirements.txt

Creamos la base de datos (sólo en el caso de que no exista)
\begin{verbatim}
python manage.py crear
python manage.py rellenar (en caso de que esté vacía)
python manage.py run
\end{verbatim}

Para el frontend, nos cambiamos la carpeta Frontend y ejecutamos:
\begin{verbatim}
npm install
npm run serve
\end{verbatim}

\subsection {Ejecución en docker}
Para la ejecución en docker sólo hay que ejecutar en el directorio raiz del código:
\begin{verbatim}
docker-compose up
\end{verbatim}

\section{Pruebas del sistema}

Para la ejecución de las pruebas unitarias del backend hay que ejecutar:
\begin{verbatim}
pytest
\end{verbatim}