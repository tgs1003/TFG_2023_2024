\apendice{Documentación técnica de programación}

\section{Introducción}

En este apéndice se va a explicar cómo está organizado el proyecto, el
manual del programador y los requisitos necesarios para poder ejecutar el
proyecto.

\section{Estructura de directorios}

En el proyecto hay una estructura de directorios que intenta organizar el trabajo de manera lógica:
Por un lado tenemos el código y por otra la documentación.
Dentro del código podemos apreciar una carpeta llamada colab con un cuaderno de Jupyter para poder ejecutarlo en Colab.
Hay otra carpeta con el código del proyecto es sí.
La estructura de directorios quedaría así:

\begin{itemize}
	\item Código
	\begin{itemize}
		\item colab: Contiene un cuaderno de Jupyter preparado para ejecutar un ejemplo en Colab.
		\item prompt\_sentiment: Contiene el código del trabajo.
		\begin{itemize}
			\item frontend: Contiene el código de la interfaz de usuario. Está hecha con Vue, usando el plugin Vuetify.
			\item backend: Contiene el código del servicio web de la aplicación. Está hecha en Flask (python) usando Flask-restx y SQLAlchemy para el acceso a las base de datos.
			\item ngix: En este directorio está la configuración del servidor ngix que se usa en la aplicación como proxy inverso y para gestionar la configuración https.
		\end{itemize}
	\end{itemize}
	\item Documentación
	\begin{itemize}
		\item Manual de usuario: El manual de uso de la aplicación
		\item Memoria: Es la memoria del proyecto
	\end{itemize}
\end{itemize}

\section{Manual del programador}

\subsection {Instalación local}
Para la instalación local del proyecto tenemos realizar las siguientes instalaciones:

\subsubsection {Instalación de Python 3 ~\cite{lutz2001programming}}
Para ejecutar nuestro proyecto es necesario instalar Python, para ello
podemos descargar la versión aquí: \url{https://www.python.org/downloads/}
Cuando la descarga haya finalizado ejecutamos e instalamos.
También se recomienda usar conda para crear un entorno virtual para python. 
\url{https://docs.conda.io/projects/conda/en/latest/index.html}

\subsubsection {Instalación de node.js}
Para la parte de angular es necesario instalar Nodejs, por tanto accederemos 
a la página oficial, \url{https://nodejs.org/es/download/}, 
y descargaremos la versión que necesitemos y llevaremos a cabo su instalación

\subsubsection {Instalación de PostgreSql}
Para la instalación de PostgreSql tenemos que bajarnos la versión adecuada a nuestro sistema operativo e instalarla:
\url{https://www.postgresql.org/download/}

\subsection {Docker}
Para la instalación de docker sólo hay que instalar Docker-Desktop.
\subsubsection {Instalación de Docker Desktop}
\url{https://docs.docker.com/engine/install/}

\section{Compilación, instalación y ejecución del proyecto}
\subsection {Ejecución local}
El proyecto tiene 2 partes claramente diferenciadas: \emph{frontend} y \emph{backend}.
Para la ejecución del backend tenemos que abrir una consola de comandos, movernos al directorio backend 
y ejecutar:
\begin{verbatim}
	pip install -r requirements.txt
\end{verbatim}

Creamos la base de datos (sólo en el caso de que no exista)
\begin{verbatim}
	python manage.py crear
	python manage.py rellenar (en caso de que esté vacía)
	python manage.py run
\end{verbatim}

Para el frontend, nos cambiamos la carpeta \emph{frontend} y ejecutamos:
\begin{verbatim}
	npm install
	npm run serve
\end{verbatim}

\subsection {Ejecución en docker}
Para la ejecución en docker sólo hay que ejecutar en el directorio raiz del código:
\begin{verbatim}
	docker-compose up
\end{verbatim}
\newpage
\section{Pruebas del sistema}

A partir de los casos de uso del anexo b, diseñamos unos casos de prueba para 
validar la aplicación desde el punto de vista de un usuario.

\subsection{Pruebas unitarias}
Para las pruebas unitarias del backend se ha utilizado la librería \emph{pytest}.
Se han realizado tanto pruebas de caja blanca como pruebas de caja negra.
Las pruebas unitarias se encuentran en el directorio \emph{backend/app/test} del repositorio.
Para la ejecución de las pruebas unitarias del \emph{backend} hay que ejecutar:
\begin{verbatim}
	pytest
\end{verbatim}
El resultado de las pruebas unitarias es:
\imagen{resultado_pytest}{Resultado pytest}

\subsection{Plan de pruebas}

\begin{table}[p]
	\centering
	\begin{tabularx}{\linewidth}{ p{0.21\columnwidth} p{0.71\columnwidth} }
		\toprule
		\textbf{CP-01}    & \textbf{Registro de usuarios}\\
		\toprule
		\textbf{Versión}              & 1.0    \\
		\textbf{Autor}                & Teodoro Ricardo García Sánchez \\
		\textbf{Requisitos asociados} & RF-1 \\
		\textbf{Descripción}          & Un usuario podrá crearse una cuenta en la aplicación.\\
		\textbf{Precondiciones}         &  
		\begin{enumerate}
			\def\labelenumi{\arabic{enumi}.}
			\tightlist
			\item No existe un usuario con la misma cuenta de correo.
			\item El formato de correo es correcto.
			\item La contraseña es válida.
		\end{enumerate}\\
		\textbf{Pasos}             &
		\begin{enumerate}
			\def\labelenumi{\arabic{enumi}.}
			\tightlist
			\item El usuario accede a la interfaz de la aplicación.
			\item Accede a la opción Registrar
			\item Introduce los datos del usuario
			\item Confirma los cambios.
		\end{enumerate}\\
		\textbf{Resultado}          & 
		\begin{enumerate}
			\item Usuario creado correctamente.
		\end{enumerate}\\
		\bottomrule
	\end{tabularx}
	\caption{CP-01 Registro de usuarios.}
\end{table}

\begin{table}[p]
	\centering
	\begin{tabularx}{\linewidth}{ p{0.21\columnwidth} p{0.71\columnwidth} }
		\toprule
		\textbf{CP-02}    & \textbf{Registro de usuarios incorrecto: el usuario ya existe}\\
		\toprule
		\textbf{Versión}              & 1.0    \\
		\textbf{Autor}                & Teodoro Ricardo García Sánchez \\
		\textbf{Requisitos asociados} & RF-1 \\
		\textbf{Descripción}          & Un usuario no puede crearse una cuenta con email existente.\\
		\textbf{Precondiciones}       &  
		\begin{enumerate}
			\def\labelenumi{\arabic{enumi}.}
			\tightlist
			\item Existe un usuario con la misma cuenta de correo.
			\item El formato de correo es correcto.
			\item La contraseña es válida.
		\end{enumerate}\\
		\textbf{Pasos}             &
		\begin{enumerate}
			\def\labelenumi{\arabic{enumi}.}
			\tightlist
			\item El usuario accede a la interfaz de la aplicación.
			\item Accede a la opción Registrar
			\item Introduce los datos del usuario
			\item Confirma los cambios.
		\end{enumerate}\\
		\textbf{Resultado}          & 
		\begin{enumerate}
			\item Usuario no creado.
			\item La aplicación devuelve un error: el usuario ya existe.
		\end{enumerate}\\
		\bottomrule
	\end{tabularx}
	\caption{CP-02 Registro de usuarios incorrecto.}
\end{table}

\begin{table}[p]
	\centering
	\begin{tabularx}{\linewidth}{ p{0.21\columnwidth} p{0.71\columnwidth} }
		\toprule
		\textbf{CP-03}    & \textbf{Registro de usuarios: email incorrecto}\\
		\toprule
		\textbf{Versión}              & 1.0    \\
		\textbf{Autor}                & Teodoro Ricardo García Sánchez \\
		\textbf{Requisitos asociados} & RF-1 \\
		\textbf{Descripción}          & El email tiene que tener un formato correcto\\
		\textbf{Precondiciones}       &  
		\begin{enumerate}
			\def\labelenumi{\arabic{enumi}.}
			\tightlist
			\item No existe un usuario con la misma cuenta de correo.
			\item El formato de correo es incorrecto.
			\item La contraseña es válida.
		\end{enumerate}\\
		\textbf{Pasos}             &
		\begin{enumerate}
			\def\labelenumi{\arabic{enumi}.}
			\tightlist
			\item El usuario accede a la interfaz de la aplicación.
			\item Accede a la opción Registrar.
			\item Introduce los datos del usuario.
			\item Confirma los cambios.
		\end{enumerate}\\
		\textbf{Resultado}          & 
		\begin{enumerate}
			\item Usuario no creado.
			\item La aplicación devuelve un error: el formato de correo no es correcto.
		\end{enumerate}\\
		\bottomrule
	\end{tabularx}
	\caption{CP-03 Registro de usuarios: email incorrecto.}
\end{table}

\begin{table}[p]
	\centering
	\begin{tabularx}{\linewidth}{ p{0.21\columnwidth} p{0.71\columnwidth} }
		\toprule
		\textbf{CP-04}    & \textbf{Registro de usuarios: contraseña no válida}\\
		\toprule
		\textbf{Versión}              & 1.0    \\
		\textbf{Autor}                & Teodoro Ricardo García Sánchez \\
		\textbf{Requisitos asociados} & RF-1 \\
		\textbf{Descripción}          & La contraseña tiene que ser válida.\\
		\textbf{Precondiciones}       &  
		\begin{enumerate}
			\def\labelenumi{\arabic{enumi}.}
			\tightlist
			\item No existe un usuario con la misma cuenta de correo.
			\item El formato de correo es correcto.
			\item La contraseña no es válida.
		\end{enumerate}\\
		\textbf{Pasos}             &
		\begin{enumerate}
			\def\labelenumi{\arabic{enumi}.}
			\tightlist
			\item El usuario accede a la interfaz de la aplicación.
			\item Accede a la opción Registrar.
			\item Introduce los datos del usuario.
			\item Confirma los cambios.
		\end{enumerate}\\
		\textbf{Resultado}          & 
		\begin{enumerate}
			\item Usuario no creado.
			\item La aplicación devuelve un error: la contraseña no es válida.
		\end{enumerate}\\
		\bottomrule
	\end{tabularx}
	\caption{CP-04 Registro de usuarios:contraseña no es válida.}
\end{table}

\begin{table}[p]
	\centering
	\begin{tabularx}{\linewidth}{ p{0.21\columnwidth} p{0.71\columnwidth} }
		\toprule
		\textbf{CP-05}    & \textbf{Registro de usuarios (administrador).}\\
		\toprule
		\textbf{Versión}              & 1.0    \\
		\textbf{Autor}                & Teodoro Ricardo García Sánchez \\
		\textbf{Requisitos asociados} & RF-1 \\
		\textbf{Descripción}          & Un usuario con el rol de administrador podrá crear cuentas de usuario y asignarles un rol. \\
		\textbf{Precondiciones}         & 
		\begin{enumerate}
			\def\labelenumi{\arabic{enumi}.}
			\tightlist
			\item El usuario creador tiene que tener el rol de administrador.
			\item No existe un usuario con la misma cuenta de correo.
			\item El formato de correo es correcto.
			\item La contraseña es válida.
		\end{enumerate}\\
		\textbf{Pasos}             &
		\begin{enumerate}
			\def\labelenumi{\arabic{enumi}.}
			\tightlist
			\item El usuario administrador accede a la interfaz de la aplicación.
			\item Accede a la opción Crear usuarios.
			\item Introduce los datos del usuarios.
			\item Confirma los cambios.
		\end{enumerate}\\
		\textbf{Resultado}          & 
		\begin{enumerate}
			\item Usuario creado correctamente.
		\end{enumerate}\\
		\bottomrule
	\end{tabularx}
	\caption{CP-05 Registro de usuarios(administrador).}
\end{table}

\begin{table}[p]
	\centering
	\begin{tabularx}{\linewidth}{ p{0.21\columnwidth} p{0.71\columnwidth} }
		\toprule
		\textbf{CP-06}    & \textbf{Registro de usuarios (administrador): usuario ya existe}\\
		\toprule
		\textbf{Versión}              & 1.0    \\
		\textbf{Autor}                & Teodoro Ricardo García Sánchez \\
		\textbf{Requisitos asociados} & RF-1 \\
		\textbf{Descripción}          & Un usuario con el rol de administrador no puede crear un usuario duplicado. \\
		\textbf{Precondiciones}         & 
		\begin{enumerate}
			\def\labelenumi{\arabic{enumi}.}
			\tightlist
			\item El usuario creador tiene que tener el rol de administrador.
			\item Existe un usuario con la misma cuenta de correo.
			\item El formato de correo es correcto.
			\item La contraseña es válida.
		\end{enumerate}\\
		\textbf{Pasos}             &
		\begin{enumerate}
			\def\labelenumi{\arabic{enumi}.}
			\tightlist
			\item El usuario administrador accede a la interfaz de la aplicación.
			\item Accede a la opción Crear usuarios.
			\item Introduce los datos del usuarios.
			\item Confirma los cambios.
		\end{enumerate}\\
		\textbf{Resultado}          & 
		\begin{enumerate}
			\item Usuario no creado.
			\item La aplicación muestra el mensaje: el usuario ya existe.
		\end{enumerate}\\
		\bottomrule
	\end{tabularx}
	\caption{CP-06 Registro de usuarios (administrador): usuario ya existe.}
\end{table}

\begin{table}[p]
	\centering
	\begin{tabularx}{\linewidth}{ p{0.21\columnwidth} p{0.71\columnwidth} }
		\toprule
		\textbf{CP-07}    & \textbf{Registro de usuarios (administrador):correo incorrecto}\\
		\toprule
		\textbf{Versión}              & 1.0    \\
		\textbf{Autor}                & Teodoro Ricardo García Sánchez \\
		\textbf{Requisitos asociados} & RF-1 \\
		\textbf{Descripción}          & Un usuario con el rol de administrador no puede crear un usuario con un correo con formato incorrecto. \\
		\textbf{Precondiciones}         & 
		\begin{enumerate}
			\def\labelenumi{\arabic{enumi}.}
			\tightlist
			\item El usuario creador tiene que tener el rol de administrador.
			\item No existe un usuario con la misma cuenta de correo.
			\item El formato de correo es incorrecto.
			\item La contraseña es válida.
		\end{enumerate}\\
		\textbf{Pasos}             &
		\begin{enumerate}
			\def\labelenumi{\arabic{enumi}.}
			\tightlist
			\item El usuario administrador accede a la interfaz de la aplicación.
			\item Accede a la opción Crear usuarios.
			\item Introduce los datos del usuarios.
			\item Confirma los cambios.
		\end{enumerate}\\
		\textbf{Resultado}          & 
		\begin{enumerate}
			\item Usuario no creado
			\item La aplicación muestra el mensaje: el correo electrónico no tiene el formato correcto.
		\end{enumerate}\\
		\bottomrule
	\end{tabularx}
	\caption{CP-07 Registro de usuarios (administrador): correo incorrecto.}
\end{table}

\begin{table}[p]
	\centering
	\begin{tabularx}{\linewidth}{ p{0.21\columnwidth} p{0.71\columnwidth} }
		\toprule
		\textbf{CP-08}    & \textbf{Registro de usuarios (administrador): contraseña no válida}\\
		\toprule
		\textbf{Versión}              & 1.0    \\
		\textbf{Autor}                & Teodoro Ricardo García Sánchez \\
		\textbf{Requisitos asociados} & RF-1 \\
		\textbf{Descripción}          & Un usuario con el rol de administrador no podrá crear cuentas de usuario con una contraseña no válida. \\
		\textbf{Precondiciones}         & 
		\begin{enumerate}
			\def\labelenumi{\arabic{enumi}.}
			\tightlist
			\item El usuario creador tiene que tener el rol de administrador.
			\item No existe un usuario con la misma cuenta de correo.
			\item El formato de correo es correcto.
			\item La contraseña no es válida.
		\end{enumerate}\\
		\textbf{Pasos}             &
		\begin{enumerate}
			\def\labelenumi{\arabic{enumi}.}
			\tightlist
			\item El usuario administrador accede a la interfaz de la aplicación.
			\item Accede a la opción Crear usuarios.
			\item Introduce los datos del usuarios.
			\item Confirma los cambios.
		\end{enumerate}\\
		\textbf{Resultado}          & 
		\begin{enumerate}
			\item Usuario no creado.
			\item La aplicación muestra el mensaje: contraseña no válida.
		\end{enumerate}\\
		\bottomrule
	\end{tabularx}
	\caption{CP-08 Registro de usuarios.}
\end{table}

\begin{table}[p]
	\centering
	\begin{tabularx}{\linewidth}{ p{0.21\columnwidth} p{0.71\columnwidth} }
		\toprule
		\textbf{CP-09}    & \textbf{Login de usuarios}\\
		\toprule
		\textbf{Versión}              & 1.0    \\
		\textbf{Autor}                & Teodoro Ricardo García Sánchez \\
		\textbf{Requisitos asociados} & RF-1 \\
		\textbf{Descripción}          & Un usuario podrá logarse en la aplicación. \\
		\textbf{Precondición}         & 
		\begin{enumerate}
			\def\labelenumi{\arabic{enumi}.}
			\tightlist
			\item Usuario no logado.
			\item El usuario existe.
			\item La contraseña es válida.
		\end{enumerate}\\
		\textbf{Pasos}             &
		\begin{enumerate}
			\def\labelenumi{\arabic{enumi}.}
			\tightlist
			\item El usuario accede a la interfaz de la aplicación.
			\item Introduce los datos de usuario.
			\item Confirma los datos.
		\end{enumerate}\\
		\textbf{Resultado}          & 
		\begin{enumerate}
			\item Usuario logado correctamente.
		\end{enumerate}\\
		\bottomrule
	\end{tabularx}
	\caption{CP-09 Login de usuarios.}
\end{table}

\begin{table}[p]
	\centering
	\begin{tabularx}{\linewidth}{ p{0.21\columnwidth} p{0.71\columnwidth} }
		\toprule
		\textbf{CP-10}    & \textbf{Login de usuario: usuario no existe}\\
		\toprule
		\textbf{Versión}              & 1.0    \\
		\textbf{Autor}                & Teodoro Ricardo García Sánchez \\
		\textbf{Requisitos asociados} & RF-1 \\
		\textbf{Descripción}          & Un usuario no podrá logarse en la aplicación. \\
		\textbf{Precondición}         & 
		\begin{enumerate}
			\def\labelenumi{\arabic{enumi}.}
			\tightlist
			\item Usuario no logado.
			\item El usuario no existe.
			\item La contraseña es indiferente.
		\end{enumerate}\\
		\textbf{Pasos}             &
		\begin{enumerate}
			\def\labelenumi{\arabic{enumi}.}
			\tightlist
			\item El usuario accede a la interfaz de la aplicación.
			\item Introduce los datos de usuario.
			\item Confirma los datos.
		\end{enumerate}\\
		\textbf{Resultado}          & 
		\begin{enumerate}
			\item Usuario no logado.
			\item La aplicación muestra el error: el usuario no existe.
		\end{enumerate}\\
		\bottomrule
	\end{tabularx}
	\caption{CP-10 Login de usuario: usuario no existe.}
\end{table}


\begin{table}[p]
	\centering
	\begin{tabularx}{\linewidth}{ p{0.21\columnwidth} p{0.71\columnwidth} }
		\toprule
		\textbf{CP-11}    & \textbf{Login de usuario: contraseña incorrecta.}\\
		\toprule
		\textbf{Versión}              & 1.0    \\
		\textbf{Autor}                & Teodoro Ricardo García Sánchez \\
		\textbf{Requisitos asociados} & RF-1 \\
		\textbf{Descripción}          & Un usuario no podrá logarse en la aplicación. \\
		\textbf{Precondición}         & 
		\begin{enumerate}
			\def\labelenumi{\arabic{enumi}.}
			\tightlist
			\item Usuario no logado.
			\item El usuario existe.
			\item La contraseña es incorrecta.
		\end{enumerate}\\
		\textbf{Pasos}             &
		\begin{enumerate}
			\def\labelenumi{\arabic{enumi}.}
			\tightlist
			\item El usuario accede a la interfaz de la aplicación.
			\item Introduce los datos de usuario.
			\item Confirma los datos.
		\end{enumerate}\\
		\textbf{Resultado}          & 
		\begin{enumerate}
			\item Usuario no logado.
			\item La aplicación muestra el error: contraseña incorrecta.
		\end{enumerate}\\
		\bottomrule
	\end{tabularx}
	\caption{CP-11 Login de usuario: contraseña incorrecta.}
\end{table}

\begin{table}[p]
	\centering
	\begin{tabularx}{\linewidth}{ p{0.21\columnwidth} p{0.71\columnwidth} }
		\toprule
		\textbf{CP-12}    & \textbf{Información de la aplicación}\\
		\toprule
		\textbf{Versión}              & 1.0    \\
		\textbf{Autor}                & Teodoro Ricardo García Sánchez \\
		\textbf{Requisitos asociados} & RF-3 \\
		\textbf{Descripción}          & Un usuario podrá acceder a la pantalla de información. \\
		\textbf{Precondición}         & El usuario tiene que estar logado. \\
		\textbf{Pasos}             &
		\begin{enumerate}
			\def\labelenumi{\arabic{enumi}.}
			\tightlist
			\item El usuario accede a la interfaz de la aplicación.
			\item El usuario hace login en la aplicación.
			\item El usuario hace clic en ``Acerca de..''.
		\end{enumerate}\\
		\textbf{Resultado}          & 
		\begin{enumerate}
			\item La aplicación muestra la pantalla de información.
		\end{enumerate}\\
		\bottomrule
	\end{tabularx}
	\caption{CP-12 Información de la aplicación.}
\end{table}

\begin{table}[p]
	\centering
	\begin{tabularx}{\linewidth}{ p{0.21\columnwidth} p{0.71\columnwidth} }
		\toprule
		\textbf{CP-13}    & \textbf{Información de la aplicación sin logarse}\\
		\toprule
		\textbf{Versión}              & 1.0    \\
		\textbf{Autor}                & Teodoro Ricardo García Sánchez \\
		\textbf{Requisitos asociados} & RF-3 \\
		\textbf{Descripción}          & Un usuario podrá acceder a la pantalla de información. \\
		\textbf{Precondición}         & El usuario no tiene que estar logado. \\
		\textbf{Pasos}             &
		\begin{enumerate}
			\def\labelenumi{\arabic{enumi}.}
			\tightlist
			\item El usuario accede a la interfaz de la aplicación.
			\item El usuario navega a ``/about''.
		\end{enumerate}\\
		\textbf{Resultado}          & 
		\begin{enumerate}
			\item La aplicación no muestra la pantalla de información.
		\end{enumerate}\\
		\bottomrule
	\end{tabularx}
	\caption{CP-13 Información de la aplicación.}
\end{table}

\begin{table}[p]
	\centering
	\begin{tabularx}{\linewidth}{ p{0.21\columnwidth} p{0.71\columnwidth} }
		\toprule
		\textbf{CP-14}    & \textbf{Analizar reseñas}\\
		\toprule
		\textbf{Versión}              & 1.0    \\
		\textbf{Autor}                & Teodoro Ricardo García Sánchez \\
		\textbf{Requisitos asociados} & RF-4 \\
		\textbf{Descripción}          & Un usuario podrá analizar un fichero de reseñas. \\
		\textbf{Precondición}         & El usuario tiene que estar logado. \\
		\textbf{Pasos}             &
		\begin{enumerate}
			\def\labelenumi{\arabic{enumi}.}
			\tightlist
			\item El usuario accede a la interfaz de la aplicación.
			\item El usuario pulsa en ``Análisis de reseñas''.
			\item El usuario carga un fichero de reseñas.
			\item El usuario configura la información del fichero de reseñas.
			\item El usuario pulsa en procesar
		\end{enumerate}\\
		\textbf{Resultado}          & 
		\begin{enumerate}
			\item La aplicación muestra la información del análisis.
		\end{enumerate}\\
		\bottomrule
	\end{tabularx}
	\caption{CP-14 Cargar reseñas.}
\end{table}

\begin{table}[p]
	\centering
	\begin{tabularx}{\linewidth}{ p{0.21\columnwidth} p{0.71\columnwidth} }
		\toprule
		\textbf{CP-15}    & \textbf{Analizar reseñas: fichero con formato incorrecto.}\\
		\toprule
		\textbf{Versión}              & 1.0    \\
		\textbf{Autor}                & Teodoro Ricardo García Sánchez \\
		\textbf{Requisitos asociados} & RF-4.1 \\
		\textbf{Descripción}          & Un usuario no podrá cargar un fichero incorrecto. \\
		\textbf{Precondición}         & El usuario tiene que estar logado. \\
		\textbf{Pasos}             &
		\begin{enumerate}
			\def\labelenumi{\arabic{enumi}.}
			\tightlist
			\item El usuario accede a la interfaz de la aplicación.
			\item El usuario pulsa en ``Análisis de reseñas''.
			\item El usuario carga un fichero de reseñas incorrecto.
		\end{enumerate}\\
		\textbf{Resultado}          & 
		\begin{enumerate}
			\item La aplicación muestra un mensaje diciendo que el fichero es incorrecto.
		\end{enumerate}\\
		\bottomrule
	\end{tabularx}
	\caption{CP-15 Cargar reseñas.: fichero con formato incorrecto.}
\end{table}

\begin{table}[p]
	\centering
	\begin{tabularx}{\linewidth}{ p{0.21\columnwidth} p{0.71\columnwidth} }
		\toprule
		\textbf{CP-16}    & \textbf{Ver información estadística}\\
		\toprule
		\textbf{Versión}              & 1.0    \\
		\textbf{Autor}                & Teodoro Ricardo García Sánchez \\
		\textbf{Requisitos asociados} & RF-4.1 \\
		\textbf{Descripción}          & Un usuario podrá ver la información estadística. \\
		\textbf{Precondición}         & El usuario tiene que estar logado. \\
		\textbf{Pasos}             &
		\begin{enumerate}
			\def\labelenumi{\arabic{enumi}.}
			\tightlist
			\item El usuario accede a la interfaz de la aplicación.
			\item El usuario pulsa en ``Histórico de reseñas''.
		\end{enumerate}\\
		\textbf{Resultado}          & 
		\begin{enumerate}
			\item El usuario puede ver la información estadística.
		\end{enumerate}\\
		\bottomrule
	\end{tabularx}
	\caption{CP-16 Ver información estadística.}
\end{table}

\begin{table}[p]
	\centering
	\begin{tabularx}{\linewidth}{ p{0.21\columnwidth} p{0.71\columnwidth} }
		\toprule
		\textbf{CP-17}    & \textbf{Perfil administrador}\\
		\toprule
		\textbf{Versión}              & 1.0    \\
		\textbf{Autor}                & Teodoro Ricardo García Sánchez \\
		\textbf{Requisitos asociados} & RF-4.1 \\
		\textbf{Descripción}          & Un usuario no administrador no podrá ver la pantalla de administración. \\
		\textbf{Precondición}         & El usuario no administrador tiene que estar logado. \\
		\textbf{Pasos}             &
		\begin{enumerate}
			\def\labelenumi{\arabic{enumi}.}
			\tightlist
			\item El usuario accede a la interfaz de la aplicación.
			\item El usuario no ve la opción de administración.
			\item El usuario navega a la dirección ``/admin''.
		\end{enumerate}\\
		\textbf{Resultado}          & 
		\begin{enumerate}
			\item El usuario no puede ver la pantalla de administración.
		\end{enumerate}\\
		\bottomrule
	\end{tabularx}
	\caption{CP-17 Cargar reseñas.}
\end{table}

\begin{table}[p]
	\centering
	\begin{tabularx}{\linewidth}{ p{0.21\columnwidth} p{0.71\columnwidth} }
		\toprule
		\textbf{CP-18}    & \textbf{Cambiar idioma de la aplicación}\\
		\toprule
		\textbf{Versión}              & 1.0    \\
		\textbf{Autor}                & Teodoro Ricardo García Sánchez \\
		\textbf{Requisitos asociados} & RF-4.1 \\
		\textbf{Descripción}          & Un usuario podrá cambiar el idioma de la aplicación. \\
		\textbf{Precondición}         & El usuario no tiene que estar logado. \\
		\textbf{Pasos}             &
		\begin{enumerate}
			\def\labelenumi{\arabic{enumi}.}
			\tightlist
			\item El usuario accede a la interfaz de la aplicación.
			\item El usuario cambia el idioma de la aplicación.
			\item El usuario hace login.
		\end{enumerate}\\
		\textbf{Resultado}          & 
		\begin{enumerate}
			\item La aplicación se mantiene en el idioma seleccionado.
		\end{enumerate}\\
		\bottomrule
	\end{tabularx}
	\caption{CP-18 Cambiar idioma de la aplicación.}
\end{table}

\begin{table}[p]
	\centering
	\begin{tabularx}{\linewidth}{ p{0.30\columnwidth} p{0.55\columnwidth}  p{0.10\columnwidth}}
		\hline
		\textbf{Caso de prueba} & \textbf{Título} & \textbf{Ok}\\ \hline
		CP-01 & Registro de usuarios & \includegraphics[width=0.5cm]{done_green.png} \\ \hline
		CP-02 & Registro de usuarios (ya existe) &\includegraphics[width=0.5cm]{done_green.png} \\ \hline
		CP-03 & Registro de usuarios (email incorrecto) &\includegraphics[width=0.5cm]{done_green.png} \\ \hline
		CP-04 & Registro de usuarios (contraseña no válida) &\includegraphics[width=0.5cm]{done_green.png} \\ \hline
		CP-05 & Registro de usuarios (administrador) &\includegraphics[width=0.5cm]{done_green.png} \\ \hline
		CP-06 & Registro de usuarios admin (ya existe)&\includegraphics[width=0.5cm]{done_green.png} \\ \hline
		CP-07 & Registro de usuarios admin (email incorrecto) &\includegraphics[width=0.5cm]{done_green.png} \\ \hline
		CP-08 & Registro de usuarios admin (contraseña no válida)&\includegraphics[width=0.5cm]{done_green.png} \\ \hline
		CP-09 & Login de usuarios&\includegraphics[width=0.5cm]{done_green.png} \\ \hline
		CP-10 & Login de usuarios (no existe)&\includegraphics[width=0.5cm]{done_green.png} \\ \hline
		CP-11 & Login de usuarios (contraseña incorrecta) &\includegraphics[width=0.5cm]{done_green.png} \\ \hline
		CP-12 & Información de la aplicación &\includegraphics[width=0.5cm]{done_green.png} \\ \hline
		CP-13 & Información de la aplicación (sin logarse)&\includegraphics[width=0.5cm]{done_green.png} \\ \hline
		CP-14 & Analizar reseñas &\includegraphics[width=0.5cm]{done_green.png} \\ \hline
		CP-15 & Analizar reseñas (formato incorrecto) &\includegraphics[width=0.5cm]{done_green.png} \\ \hline
		CP-16 & Ver información estadística &\includegraphics[width=0.5cm]{done_green.png} \\ \hline
		CP-17 & Perfil administrador&\includegraphics[width=0.5cm]{done_green.png} \\ \hline
		CP-18 & Idioma de la aplicación&\includegraphics[width=0.5cm]{done_green.png} \\ \hline
	\end{tabularx}
	\caption{Resultados}
\end{table}
