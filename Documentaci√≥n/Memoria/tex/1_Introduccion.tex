\capitulo{1}{Introducción}

El análisis de sentimientos intenta descubrir la actitud de un usuario 
con respecto a algún tema.
Esta puede ser una reseña, un estado afectivo o emocional. 
Este de análisis se denomina minería de opinión.

En principio el objetivo del análisis de sentimientos es 
clasificar la polaridad del texto (positiva o negativa).
Existen muchas formas de llevar a cabo este análisis (localización de palabras, 
afinidad léxica, métodos estadísticos o tecnicas conceptuales)~\cite{Bannister2015}

En este trabajo vamos a evaluar la utilización del ``\emph{Prompt Engineering}'' 
asociado a un modelo grande de lenguaje (LLM) para ese análisis de sentimiento.
Evaluaremos su viabilidad y sus resultados. 

Utilizaremos un dataset etiquetado de opiniones de clientes de Amazon para evaluar 
el rendimiento del modelo. En particular, usaremos 2 medidas, por un lado la precisión 
que mide el ratio de verdaderos positivos entre todos los que se han clasificado como tal.
y la exhaustividad (``\emph{recall}'') que mide la relación entre los verdaderos positivos y 
todos los positivos.

Se considera un modelo equiparable a un humano si consigue un 65 por ciento de aciertos.~\cite[p.~4]{Saif2013}
\newpage
\section{Estructura de la memoria}
\begin{description}
	\item[Introducción:] descripción del tema sobre el que trata el trabajo y su situación. 
    Incluye su estructura (memoria y anexos)
	\item[Objetivos del trabajo:] este apartado explica de forma concisa cuales son los objetivos 
    que se persiguen con la realización del proyecto. 
    Se pueden distinguir entre los objetivos marcados por los requisitos del software a construir y
     los objetivos de carácter técnico que plantea a la hora de llevar a la práctica el proyecto.
    \item[Conceptos teóricos:] exposición de conceptos que facilitan la comprensión del proyecto.
    \item[Técnicas y herramientas:] listado de metodologías y herramientas que han sido 
    utilizadas para llevar a cabo el proyecto.
    \item[Aspectos relevantes:] muestra aspectos a destacar durante la realización del proyecto.
    \item[Trabajos relacionados:] estado del arte en el ámbito del ``sentiment analysis'' y trabajos similares.
    \item[Conclusiones y líneas de trabajo futuras:] conclusiones obtenidas al final del proyecto y posibles ideas futuras.
\end{description}
\section{Estructura de los anexos}
\begin{description}
	\item[Plan de proyecto software:] planificación temporal y viabilidad económica y legal.
	\item[Especificación de requisitos:] objetivos y y requisitos establecidos al comienzo del proyecto.    
    \item[Especificación de diseño:] recoge los diseños de datos, procedimental, arquitectónico y de interfaces.
    \item[Manual del programador:] explica los conceptos más técnicos del proyecto como su instalación, la organización de carpetas y la ejecución.
    \item[Manual de usuario:] es la guía de cómo utilizar la aplicación paso a paso.
    \item[Sostenibilización curricular:] reflexión personal sobre los aspectos de la sostenibilidad que se abordan en el trabajo.
\end{description}
