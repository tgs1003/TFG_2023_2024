\capitulo{1}{Introducción}

El análisis de sentimientos intenta descubrir la actitud de un usuario 
con respecto a algún tema.
Esta puede ser una reseña, un estado afectivo o emocional. 
Este de análisis se denomina minería de opinión.

En principio el objetivo del análisis de sentimientos es 
clasificar la polaridad del texto (positiva o negativa).
Existen muchas formas de llevar a cabo este análisis (localización de palabras, 
afinidad léxica, métodos estadísticos o tecnicas conceptuales)~\cite{Bannister2015}

En este trabajo vamos a evaluar la utilización del ``\emph{Prompt Engineering}'' 
asociado a un modelo grande de lenguaje (LLM) para ese análisis de sentimiento.
Evaluaremos su viabilidad y sus resultados. 

Usaremos dos medidas para evaluar el modelo, 
por un lado la precisión (número de elementos entre el número total) 
y la exhaustividad (recall) que mide la cantidad de elementos 
que ha sido capaz de identificar 
(se calcula dividiendo el número de aciertos entre el número 
total que hemos clasificado de esa forma). 

Se considera un modelo equiparable a un humano si consigue un 70 por ciento de aciertos.~\cite{Saif2013}

\section{Estructura de la memoria}
\begin{description}
	\item[Introducción:] Descripción del tema sobre el que trata el trabajo y su situación. 
    Incluye su estructura (memoria y anexos)
	\item[Objetivos del trabajo:] Este apartado explica de forma concisa cuales son los objetivos 
    que se persiguen con la realización del proyecto. 
    Se pueden distinguir entre los objetivos marcados por los requisitos del software a construir y
     los objetivos de carácter técnico que plantea a la hora de llevar a la práctica el proyecto.
    \item[Conceptos teóricos:] Exposición de conceptos que facilitan la comprensión del proyecto.
    \item[Técnicas y herramientas:] Listado de metodologías y herramientas que han sido 
    utilizadas para llevar a cabo el proyecto.
    \item[Aspectos relevantes:] Muestra aspectos a destacar durante la realización del proyecto.
    \item[Trabajos relacionados:] Estado del arte en el ámbito del ``sentiment analysis'' y trabajos similares.
    \item[Conclusiones y líneas de trabajo futuras:] Conclusiones obtenidas al final del proyecto y posibles ideas futuras.
\end{description}
\section{Estructura de los anexos}
\begin{description}
	\item[Plan de proyecto software:] Planificación temporal y viabilidad económica y legal.
	\item[Especificación de requisitos:] Objetivos y y requisitos establecidos al comienzo del proyecto.    
    \item[Especificación de diseño:] Recoge los diseños de datos, procedimental, arquitectónico y de interfaces.
    \item[Manual del programador:] Explica los conceptos más técnicos del proyecto como su instalación, la organización de carpetas y la ejecución.
    \item[Manual de usuario:] Es la guía de cómo utilizar la aplicación paso a paso.
    \item[Sostenibilización curricular:] Reflexión personal sobre los aspectos de la sostenibilidad que se abordan en el trabajo.
    
\end{description}
