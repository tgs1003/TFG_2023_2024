\capitulo{2}{Objetivos del proyecto}

El objetivo del trabajo propuesto es desarrollar una aplicación 
basada en inteligencia artificial (IA) utilizando modelos de lenguaje grandes (LLM~\cite{mirchandani2023large}) 
para el análisis, clasificación y calificación de reseñas de usuarios en portales 
de recomendación y tiendas de comercio electrónico. 
Los objetivos específicos son los siguientes:
\begin{itemize}
    \item[-] Crear una aplicación que utilice modelos de lenguaje grandes (como \href{https://chat.openai.com}{ChatGPT}, \href{https://bard.google.com/?hl=es}{Bard} o \href{https://research.facebook.com/publications/llama-open-and-efficient-foundation-language-models}{Llama})
    de manera efectiva para analizar y comprender las reseñas de usuarios. 
    \item[-] Diseñar un sistema basado en "prompts", definiendo las tareas, entradas y 
    salidas de la respuesta a una instrucción, para lograr una solución 
    ágil sin necesidad de nuevos entrenamientos.
    \item[-] Identificar y categorizar aspectos específicos del producto o servicio:
    \begin{description}
        \item[-] Implementar un sistema que pueda identificar y categorizar 
    los aspectos específicos mencionados por los usuarios en sus reseñas.
        \item[-] Extraer información relevante sobre productos o servicios, 
    identificando características específicas que generan opiniones.
    \end{description}
    \item[-] Relacionar opiniones y emociones con calificaciones:
    \begin{description}
        \item[-] Desarrollar un mecanismo que pueda relacionar las opiniones expresadas 
        en las reseñas con las calificaciones otorgadas por los usuarios.
        \item[-] Analizar las emociones asociadas a las opiniones para comprender 
        mejor la percepción de los usuarios respecto a productos y servicios.
    \end{description}
    \item[-] Implementar técnicas de ``\emph{prompt engineering}'' ~\cite{white2023prompt}:
    \begin{description}
        \item[-] Utilizar técnicas de ``\emph{prompt engineering}'' para optimizar la 
        interacción con el modelo de lenguaje, mejorando la calidad y 
        relevancia de las respuestas generadas.
        \item[-] Facilitar un desarrollo e implementación rápida y de bajo código para la solución propuesta. 
    \end{description}
        
\end{itemize}

El objetivo general es aprovechar la capacidad de los LLM 
para comprender el lenguaje natural y aplicarlos de manera efectiva en el 
análisis de reseñas de usuarios, proporcionando una solución ágil y 
eficiente para extraer información valiosa sobre la 
percepción de los usuarios en portales de recomendación y tiendas en línea.