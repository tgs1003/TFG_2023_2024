\capitulo{4}{Técnicas y herramientas}

En esta parte de la memoria voy a presentar algunas de las técnicas de desarrollo 
que he utilizado para el desarrollo del proyecto, así como la información relevante 
de cada una de ellas.

\section{Técnicas}

\subsection{ORM~\cite{fowler2012patterns}}
\label{ORM}
Un ORM (Object-Relational Mapping) es una técnica de programación que se 
utiliza en el desarrollo de software para conectar objetos en un programa 
con tablas en una base de datos relacional. \\
Su propósito principal es permitir a los desarrolladores trabajar con datos de una base de datos relacional utilizando objetos y clases en lugar de escribir consultas SQL directamente. Un ORM mapea las estructuras de datos en una base de datos a objetos en un lenguaje de programación, lo que facilita la interacción con la base de datos y el manejo de datos de una manera más orientada a objetos.
Las ventajas clave de usar un ``\emph{ORM}'' incluyen:
\begin{itemize}
    \item \textbf{Abstracción de la base de datos}: Oºculta los detalles de la base de datos subyacente, 
    permitiendo a los desarrolladores interactuar con los datos en términos de objetos y clases en lugar de preocuparse por la sintaxis SQL específica de la base de datos.
    \item \textbf{Productividad}: Reduce la cantidad de código SQL que los desarrolladores deben escribir, 
    lo que ahorra tiempo y esfuerzo en el desarrollo de aplicaciones.
    \item \textbf{Portabilidad}: Puede ser diseñado para ser independiente de la base de datos, 
    lo que facilita la migración de una base de datos a otra sin cambiar el código de la aplicación.
    \item \textbf{Mantenibilidad}: Al trabajar con objetos y clases en lugar de consultas SQL directas, 
    el código puede ser más fácil de mantener y entender.
    \item \textbf{Seguridad}: Suelen ofrecer características de seguridad integradas, 
    como la prevención de inyecciones SQL, lo que reduce la vulnerabilidad de las aplicaciones a ataques. 
\end{itemize}
Ejemplos de ORMs populares incluyen Hibernate para Java, Entity Framework para .NET, 
SQLAlchemy para Python, Sequelize para Node.js y Django ORM para aplicaciones web Django en Python. \\
Cada uno de estos ORMs proporciona una forma de mapear objetos a tablas de base de datos y facilita 
la manipulación de datos en aplicaciones de software de manera más eficiente y mantenible.

\subsection{MVC~\cite{fowler2012patterns}}
MVC es un patrón arquitectónico comúnmente utilizado en el desarrollo de aplicaciones de software, 
especialmente en aplicaciones web. \\
Las siglas ``\emph{MVC}'' representa ``\emph{Model-View-Controller}'', y se divide en tres componentes principales 
que desempeñan roles específicos en la organización y estructura de una aplicación. 
Estos son:
\begin{itemize}
    \item \textbf{Model (Modelo)}: El modelo representa la lógica empresarial y los datos subyacentes 
    de la aplicación.\\ En otras palabras, el modelo se encarga de manejar la manipulación y 
    el acceso a los datos, así como de realizar cálculos y procesos necesarios.\\ 
    El modelo no tiene conocimiento de la interfaz de usuario y no se preocupa por cómo se muestran 
    los datos.\\ Es independiente de la vista y el controlador.
    \item \textbf{View (Vista)}: La vista se encarga de la presentación y la interfaz de usuario. \\
    Su función principal es mostrar los datos al usuario de una manera comprensible y atractiva. \\
    La vista no realiza operaciones comerciales o de lógica, sino que simplemente muestra la 
    información proporcionada por el modelo. \\
    En sistemas web, esto a menudo se traduce en la generación de páginas HTML o 
    interfaces de usuario dinámicas.
    item \textbf{Controller (Controlador)}: El controlador actúa como un intermediario entre el modelo y la vista. 
\end{itemize}

Recibe las solicitudes del usuario a través de la vista, procesa esas solicitudes, 
realiza las operaciones necesarias en el modelo y luego actualiza la vista para mostrar los resultados.\\
El controlador maneja la lógica de interacción y toma decisiones sobre qué acción realizar 
en función de la entrada del usuario.\\
La arquitectura MVC promueve una clara separación de preocupaciones, lo que facilita la modularidad y la escalabilidad de las aplicaciones. Esto permite a los desarrolladores trabajar de manera más eficiente, ya que pueden centrarse en una capa (modelo, vista o controlador) sin preocuparse demasiado por las otras capas. Además, el MVC es ampliamente utilizado en marcos de desarrollo web como Ruby on Rails, Django (Python), Laravel (PHP), y ASP.NET MVC (CSharp), entre otros.
En resumen, el patrón MVC es una forma efectiva de organizar y estructurar aplicaciones de software, lo que resulta en un código más claro, mantenible y flexible.

\subsection{LLM~\cite{mirchandani2023large}}
\label{LLM}
Un LLM (Modelo de Lenguaje Grande) es un tipo de modelo de inteligencia artificial
 diseñado para procesar y generar texto en lenguaje natural a gran escala.\\
Estos modelos están entrenados en enormes cantidades de datos de texto 
y están diseñados para comprender y generar texto en varios idiomas, 
lo que les permite llevar a cabo tareas de procesamiento de lenguaje natural de manera efectiva. \\
Algunos ejemplos son GPT-3, GPT-4 y BERT~\cite{BertSentimentModel1}.\\
Características clave:
\begin{itemize}
    \item \textbf{Tamaño y complejidad}: los modelos de lenguaje grandes contienen miles de millones o 
    incluso decenas de miles de millones de parámetros, lo que les permite capturar una 
    gran cantidad de conocimiento lingüístico y patrones de lenguaje.
    \item \textbf{Pre-entrenamiento y afinación}: estos modelos se entrenan primero en grandes corpus de texto, lo que les permite aprender el lenguaje natural y su estructura. Luego, se pueden afinar o ajustar para tareas específicas, como traducción automática, resumen de texto, generación de texto creativo, entre otros.
    \item \textbf{Generación de texto}: son capaces de generar texto coherente y contextualmente relevante.\\
    Pueden responder preguntas, completar oraciones, generar contenido original 
    y realizar tareas de procesamiento de lenguaje natural.
    \item \textbf{Amplia aplicabilidad}: estos modelos se utilizan en una amplia gama de aplicaciones, 
    que van desde asistentes virtuales y chatbots hasta motores de búsqueda mejorados, 
    análisis de sentimientos, resumen automático de texto, traducción automática y más.
\end{itemize}
 
Plantean desafíos éticos y de seguridad, ya que su 
capacidad para generar texto implica la necesidad de prevenir el uso indebido, 
la desinformación y el sesgo en el contenido generado.\\
Estos modelos han sido desarrollados y perfeccionados por empresas de tecnología y 
organizaciones de investigación en inteligencia artificial, y han demostrado 
un gran potencial en una variedad de aplicaciones. \\
A medida que continúan evolucionando, su influencia 
en la forma en que interactuamos con la tecnología y generamos contenido 
en línea también está en constante crecimiento.


\section{Herramientas}

\subsection{Python~\cite{lutz2001programming}}
Es un lenguaje de programación de alto nivel, interpretado y generalmente 
considerado como fácil de aprender y de utilizar. \\
Python destaca por su sintaxis clara y fácil de entender, 
lo que lo convierte en un lenguaje ideal tanto para principiantes 
como para programadores experimentados.\\
Es compatible con una amplia gama de sistemas operativos, 
lo que lo hace muy portable.\\

Ofrece una extensa biblioteca estándar que cubre una variedad de tareas, 
desde manipulación de archivos y redes hasta desarrollo web y 
matemáticas, lo que acelera el desarrollo de aplicaciones.\\
Cuenta con una comunidad de desarrollo muy activa y 
una gran cantidad de bibliotecas y marcos de trabajo de código abierto disponibles, 
lo que facilita la creación de aplicaciones complejas.\\
Es orientado a objetos, lo que significa que se enfoca en la creación de objetos 
que pueden contener datos y funciones.

Se utiliza en una amplia variedad de campos, como desarrollo web, 
análisis de datos, inteligencia artificial, automatización de tareas, desarrollo de juegos y más.

\subsection{Flask~\cite{dwyer2017flask}}
Flask es un framework de desarrollo web en Python que se utiliza 
para crear aplicaciones web de manera sencilla y rápida. \\
Está dentro de lo que se denomina ``\emph{microframework}'' por su enfoque minimalista, proporciona 
las funcionalidades esenciales pero permite agregar las bibliotecas necesarias según los requisitos.\\
No impone estructuras rígidas ni componentes innecesarios pero incorpora funcionalidades 
(como el motor de plantillas Jinja2 o el sistema de rutas) que hacen 
más sencillo el desarrollo de aplicaciones. \\
En este proyecto lo utilizamos para realizar un \href{https://aws.amazon.com/es/what-is/restful-api}{API RESTful}, que es un uso bastante común de este framework.


\subsection{SqlAlchemy~\cite{bayer2012sqlalchemy}}
SQLAlchemy es una biblioteca de Python que proporciona un conjunto de herramientas y
 un ORM(\ref{ORM}) que permite a los desarrolladores interactuar 
con bases de datos relacionales de una manera más sencilla y orientada a objetos.\\

Ofrece una amplia gama de métodos y operadores para realizar consultas en la base de datos, 
lo que facilita la construcción de consultas complejas de manera programática.

Gestiona las transacciones de manera transparente, lo que asegura la integridad 
de los datos al realizar operaciones de lectura y escritura.\\
Se pueden definir relaciones entre objetos y tablas de base de datos de manera sencilla,
 lo que simplifica la representación de datos relacionados.\\
Proporciona herramientas para realizar migraciones de bases de datos, 
lo que facilita la actualización de la estructura de la base de datos a medida 
que evoluciona la aplicación.\\
Permite la creación de extensiones y complementos personalizados 
para adaptarse a necesidades específicas.\\
Se utiliza principalmente en aplicaciones web, 
especialmente en el desarrollo de aplicaciones basadas en frameworks como Flask y Django. \\
Permite a los desarrolladores trabajar con bases de datos de manera más eficiente y 
mantener un código más limpio y legible al proporcionar una capa de abstracción entre la 
aplicación y la base de datos subyacente.\\ 
Esto también facilita la portabilidad de la aplicación entre diferentes sistemas de 
gestión de bases de datos.\\

\subsection{LlamaCpp~\cite{metallamaLM}}
La división de inteligencia artificial de Meta, llamada Meta AI, presentó LLaMA, un LLM (\ref{LLM}) de 
65.000 millones de parámetros que permite disfrutar de un motor de 
IA conversacional muy parecido a ChatGPT.\\
Este modelo estaba inicialmente disponible para desarrolladores e investigadores 
que justificaran su uso.\\
Grigori Gerganov publicó en Github~\cite{gerganovllamaCpp} un pequeño desarrollo llamado llama.cpp,
un proyecto que permite poder usar el modelo LLaMA en un portátil o un PC convencional.\\ 
Eso se logra gracias a la llamada ``\emph{cuantización}'' que reduce el tamaño de los modelos de 
Facebook para hacerlos ``manejables'' por equipos más modestos a nivel de hardware.

\subsection{PostgreSql~\cite{juba2015learning}}
PostgreSQL es un sistema de gestión de bases de datos relacional de código abierto. 
Admite consultas complejas y procedimientos almacenados.\\
Proporciona mecanismos para garantizar la integridad de los datos almacenados, 
como restricciones de clave primaria y foránea, así como comprobaciones de restricciones.\\
Es escalable y se puede utilizar en aplicaciones de todos los tamaños, 
desde aplicaciones pequeñas hasta sistemas empresariales de alto rendimiento.\\
Admite la replicación y puede configurarse para lograr alta disponibilidad mediante 
la implementación de réplicas y clústeres.
Soporta autenticación, autorización y cifrado de datos.\\
PostgreSQL es una opción popular tanto en la comunidad de código abierto como 
en empresas que buscan una base de datos de alto rendimiento y confiabilidad para sus aplicaciones.\\
Su licencia de código abierto permite su uso y distribución sin costos de licencia, 
lo que lo convierte en una opción atractiva para una amplia variedad de proyectos.

\subsection{Docker~\cite{garzas2105docker}}
Docker es una plataforma que permite empaquetar una aplicación y 
todas sus dependencias en un contenedor.\\
Este contenedor es como una unidad autónoma que puede ejecutarse de manera 
consistente en cualquier entorno que admita Docker. 
Además, Docker facilita la gestión y la implementación de estas aplicaciones empaquetadas.
Al utilizar Docker, los desarrolladores pueden estar seguros de que la aplicación 
se ejecutará de la misma manera en todas partes, desde el entorno de desarrollo hasta el de producción. 
Esto elimina problemas relacionados con las diferencias entre configuraciones y facilita las pruebas.
Se pueden iniciar y detener rápidamente, lo que permite una implementación más rápida de 
nuevas versiones o características. \\
Esto es crucial en el \emph{desarrollo ágil}, donde la velocidad de respuesta a los cambios es esencial.

Además, encapsula todas las dependencias de la aplicación en el contenedor, 
evitando conflictos con otras aplicaciones o componentes del sistema. 
Esto simplifica la gestión de dependencias y reduce los posibles problemas de compatibilidad. \\
También facilita la escalabilidad horizontal, lo que significa que puedes 
ejecutar múltiples instancias de contenedores de la misma aplicación para 
gestionar cargas de trabajo más pesadas de manera eficiente.

Los contenedores Docker pueden compartirse fácilmente, lo que facilita la colaboración 
entre equipos de desarrollo y operaciones. 
Todos trabajan con la misma configuración, lo que reduce la posibilidad de errores 
causados por diferencias en los entornos.

\subsection{ChatGPT~\cite{chatgpt1}}
ChatGPT es un modelo de lenguaje desarrollado por OpenAI, basado en la arquitectura GPT (Generative Pre-trained Transformer). La sigla "GPT" hace referencia a "Generative Pre-trained Transformer", y la adición de "Chat" indica que el modelo está particularmente diseñado para tareas de conversación o diálogo.

La arquitectura GPT es una red neuronal de tipo transformer que ha sido preentrenada en grandes cantidades de datos textuales para aprender patrones y estructuras del lenguaje. Esto significa que el modelo ha sido expuesto a una amplia variedad de texto antes de ser afinado para tareas específicas. GPT-3.5 es una versión avanzada de esta arquitectura.

ChatGPT puede generar respuestas contextualmente relevantes en función de las entradas que recibe. Puede ser utilizado para una variedad de aplicaciones, desde responder preguntas y ayudar en la redacción de textos hasta generar conversaciones simuladas.

Es importante tener en cuenta que ChatGPT no posee conocimiento del mundo en tiempo real y su información se basa en datos previos a su fecha de corte en enero de 2022. Además, aunque es poderoso y versátil, no tiene la capacidad de razonamiento ni comprensión como lo haría un ser humano.

ChatGPT puede ayudarte a crear una memoria de un proyecto proporcionándote información, sugerencias, y estructurando la información de manera coherente. Aquí hay algunos pasos que puedes seguir para aprovechar la ayuda de ChatGPT en la creación de la memoria de tu proyecto:

Definir el Alcance del Proyecto:

Proporcióname detalles sobre el proyecto, como su propósito, objetivos y alcance.
Especifica las características clave y los requisitos del proyecto.
Estructura de la Memoria:

Pide ayuda para estructurar la memoria. Por ejemplo, "¿Cómo puedo organizar la información sobre la introducción, objetivos, metodología, resultados, y conclusiones?"
Redacción de Secciones Específicas:

Proporcióname detalles específicos sobre ciertas secciones. Por ejemplo, "Proporcióname información relevante para la sección de resultados."
Revisión de Contenido:

Pide ayuda para revisar y mejorar el contenido de la memoria. Puedo sugerir rephrasing, añadir detalles o mejorar la claridad.
Preguntas Detalladas:

Si hay áreas específicas en las que necesitas más claridad, puedes hacer preguntas detalladas para obtener información específica.
Consejos de Estilo y Presentación:

Solicita consejos sobre el estilo de redacción y presentación para hacer que la memoria sea más efectiva y profesional.
Inclusión de Datos Relevantes:

Si tienes datos específicos que necesitas incluir, proporciona los detalles y pregunta cómo integrarlos de manera efectiva.
Ayuda con Citas o Referencias:

Si necesitas citar fuentes o incluir referencias, pide orientación sobre cómo hacerlo correctamente.
Revisiones y Correcciones:

Solicita revisiones y correcciones gramaticales para garantizar la precisión y calidad del documento final.
