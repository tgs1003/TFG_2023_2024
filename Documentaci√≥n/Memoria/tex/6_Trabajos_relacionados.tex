\capitulo{6}{Trabajos relacionados}

\section{Lingmotif~\cite{lingmotif}}
Lingmotif es una aplicación multi-plataforma de sobremesa que analiza textos desde la perspectiva 
del Análisis de Sentimiento. \\Básicamente, es capaz de determinar la orientación semántica 
(si es positivo o negativo y en qué grado) de un texto o conjunto de textos, mediante \textbf{la detección 
de expresiones lingüísticas} que indican una determinada polaridad.

A diferencia de la mayoría del software existente, Lingmotif no es un sólo un clasificador, 
ya que no se limita a clasificar un texto como positivo o negativo, sino que además ofrece una 
serie de datos cuantitativos, una visualización del ``perfil de sentimiento'' del texto o textos 
(incluyendo series temporales), y un detallado análisis cualitativo del texto en sí, en el que 
se muestran los segmentos textuales identificados. \\Estas funcionalidades lo convierten en un herramienta 
única, y sus aplicaciones van más allá de las que normalmente ofrece este tipo de software. 
Lingmotif ofrece los resultados de sus análisis en archivos con formato HTML, con la versatilidad y 
fácil manejo que esto supone.

Actualmente Lingmotif analiza textos en español e inglés. Se está trabajando en nuevas versiones 
para alemán e italiano.

\section{\href{https://www.lexalytics.com/}{Lexalytics}}
Lexalytics es una empresa que se especializa en tecnologías de análisis de texto y 
procesamiento del lenguaje natural (PLN).\\
Ofrece soluciones para extraer información de datos de texto no estructurados, 
como contenido de redes sociales, comentarios de clientes, reseñas y más. 
Los productos de Lexalytics están diseñados para ayudar a las empresas a analizar 
y comprender el sentimiento, los temas y las tendencias dentro de grandes volúmenes de información textual.
Uno de sus productos destacados es la Plataforma de Inteligencia Lexalytics, 
que incluye varias herramientas y funciones para el análisis de texto. 
Tienen un API (al que llaman Semantria) que se usa para análisis de texto y sentimiento.
Admite varios idiomas, lo que las hace adecuadas para empresas con presencia global.\\
Estas herramientas y servicios pueden ser valiosos para empresas que 
buscan dar sentido a la gran cantidad de texto no estructurado que generan o 
encuentran en el curso de sus operaciones. \\Se pueden aplicar en áreas como la gestión 
de la experiencia del cliente, el monitoreo de redes sociales, la investigación de mercado 
y otros campos donde comprender la información textual es crucial. 

\section{\href{https://www.ibm.com/products/watson-studio}{IBM Watson Studio}}
IBM Watson Studio es una plataforma de inteligencia artificial (IA) desarrollada 
por IBM que proporciona servicios y herramientas para el procesamiento 
del lenguaje natural y otros. 
Lleva el nombre del fundador de IBM (Thomas J. Watson)
También utiliza técnicas de aprendizaje automático 
para extraer patrones a partir de grandes conjuntos de datos y realizar análisis predictivos.\\
Tiene capacidades para analizar y comprender imágenes y videos, lo que puede 
ser útil en aplicaciones como el reconocimiento facial y la interpretación de contenido visual.\\
Se ha utilizado en aplicaciones como juegos de preguntas y 
respuestas, demostrando su capacidad para competir y ganar contra humanos en contextos complejos.
Puede automatizar tareas y procesos empresariales, mejorando la eficiencia operativa.
Analiza datos en tiempo real para proporcionar información instantánea y tomar decisiones basadas en datos.
IBM Watson se ha aplicado en diversos campos obtenido muy buenos resultados.
Permite crear un cuanta gratuita (limitada) para poder probarla.

\section{Sentinel}
Sentinel~\cite{Sentinel1} es una aplicación web desarrollada como TFG del grado de ingeniería 
informática de la Universidad de Burgos que realiza un análisis de sentimientos de textos 
extraídos de las redes sociales Twitter e Instagram.\\
La aplicación busca los tweets relacionados con la palabra introducida 
en el caso de Twitter, o los comentarios que le han escrito a la 
cuenta de Instagram que se ha buscado.\\
Después analiza el sentimiento que hay en ellos, los puntúa con valores entre 0 y 1 y 
se almacenan los resultados.\\
Estos resultados se muestran al usuario en gráficos 
y tablas para hacer la experiencia más visual.\\ 
Además se le ofrece la opción de calcular series temporales a partir de los resultados, 
y se da una predicción de los valores futuros.

\section{\href{https://www.danielsoper.com/sentimentanalysis/default.aspx}{Free Sentiment Analysis}}
Es una herramienta gratuita que permite realizar un analisis de sentimiento de cualquier 
texto escrito en Inglés. La herramienta califica el sentimiento con un número entre -100 y 100 
dependiendo del sentimiento detectado en el texto.
Sólo hay que pegar el texto en el cuadro de texto y pulsar el botón ``Analyze text''

Para su funcionamiento utiliza algoritmos de linguística computacional y minería de texto.
El modelo se ha entrenado usando el ``American National Corpus'' con lo que sólo funciona con 
textos en inglés americano y escritos después de 1990. 
No se conoce su funcionamiento interno pero por la descripción parece que ha usado un 
corpus anotado y usa las coincidencias de palabras para dar una nota al texto dependiendo 
de las palabras encontradas.