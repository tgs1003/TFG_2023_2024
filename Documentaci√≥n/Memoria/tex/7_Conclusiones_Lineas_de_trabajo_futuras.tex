\capitulo{7}{Conclusiones y Líneas de trabajo futuras}

\section{Conclusiones}


\section{Líneas de trabajo futuras}

\subsection{Nuevos modelos}
Actualmente la aplicación soporta el uso de OpenAI y Hugging face debido a las limitaciones de ambas,
sería adecuado ampliar las opciones. Una de las posible opciones sería instalar un modelo en local 
pero se descartó debido al pobre rendimiento y la excesiva cantidad de recursos necesarios.
La opción más viable sería incluir un gestor de colas de mensajes (tipo RabbitMQ) para procesar las distintas tareas.
De esta forma se podría distribuir el trabajo en diferentes ordenadores con mejores prestaciones 
(incluyendo tarjetas gráficas).

\subsection{Nuevas fuentes de datos}
La aplicación en el momento de presentación al tribunal sólo permite procesar reseñas desde un fichero (ya sea Csv o Json).
Sería interesante agregar nuevas fuentes de datos como servicios web o similares.

\subsection{Más opciones de análisis}
Sería muy interesante darle a usuario la posibilidad de modificar el prompt para permitir 
extraer más información útil de las reseñas.

\subsection{GridFS}
Ahora mismo al importar un fichero de reseñas éste se almacena en directorio temporal en el contenedor docker.\\
Podría ser interesante utilizar una solución más robusta como puede ser \href{https://www.mongodb.com/docs/manual/core/gridfs/}{GridFS}. 
Aunque una de las ventajas de usarlo en un directorio temporal es que estos ficheros desaparecen al reiniciar.
