\capitulo{7}{Conclusiones y Líneas de trabajo futuras}

\section{Conclusiones}
El uso de modelos grandes de lenguaje está cada vez más extendido en muchos sectores tecnológicos.\\
En estos momentos hay una explosión de proyectos y usos que no eramos capaces de imaginarnos hace un años.\\
En este trabajo se han combinado estas técnicas con otras más tradicionales para 
conseguir una aplicación útil con muchas oportunidades de mejora y ampliación.\\
Lo que he intentado es establecer una plataforma que sea flexible para poderla ampliar y 
escalar tanto como se necesite.\\
En el camino he aprendido desde el uso de LLMs, python, web services, seguridad (mediante jwts), 
despliegue continuo con docker, etc.
El resultado creo que es una aplicación sencilla pero con mucho potencial y suficientemente 
flexible como para servir de base de otros desarrollos más complejos.
\newpage

\section{Líneas de trabajo futuras}

\subsection{Nuevos modelos}
Actualmente la aplicación soporta el uso de OpenAI y Hugging face debido a las limitaciones de ambas,
sería adecuado ampliar las opciones. Una de las posible opciones sería instalar un modelo en local 
pero se descartó debido al pobre rendimiento y la excesiva cantidad de recursos necesarios.
La opción más viable sería incluir un gestor de colas de mensajes (tipo RabbitMQ) para procesar las distintas tareas.
De esta forma se podría distribuir el trabajo en diferentes ordenadores con mejores prestaciones 
(incluyendo tarjetas gráficas).

\subsection{Nuevas fuentes de datos}
La aplicación en el momento de presentación al tribunal sólo permite procesar reseñas desde un fichero (ya sea Csv o Json).
Sería interesante agregar nuevas fuentes de datos como servicios web o similares.

\subsection{Más opciones de análisis}
Sería muy interesante darle a usuario la posibilidad de modificar el prompt para permitir 
extraer más información útil de las reseñas.

\subsection{GridFS}
Ahora mismo al importar un fichero de reseñas éste se almacena en directorio temporal en el contenedor docker.\\
Podría ser interesante utilizar una solución más robusta como puede ser \href{https://www.mongodb.com/docs/manual/core/gridfs/}{GridFS}. 
Aunque una de las ventajas de usarlo en un directorio temporal es que estos ficheros desaparecen al reiniciar.
