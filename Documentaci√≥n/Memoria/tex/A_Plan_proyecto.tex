\apendice{Plan de Proyecto Software}

\section{Introducción}
Este primer apartado de los anexos se ha dedicado a la explicación del
desarrollo del proyecto, recogido en planificación temporal, y a la viabilidad
del proyecto, dividida en viabilidad económica donde se hará una aproximación de los costes que conlleva el proyecto y viabilidad legal donde se debe
tener en cuenta la legislación que puede afectar al proyecto.

\section{Planificación temporal}
Antes de comenzar el proyecto se decidió utilizar metodología ágil para
gestionarlo. Se han realizado sprints de dos semanas de duración.
Al final de cada sprint se ha hecho revisión para comprobar si se habían
conseguido los objetivos marcados al inicio y plantear las tareas a
realizar para el nuevo sprint.
Para la gestión del proyecto se han utilizado GitHub y Zube. El
tablero que provee Zube ha permitido que visualmente la organización
del proyecto haya sido más sencilla. Desde github se han vinculado los ``commits'' 
con cada tarea para hacer más fácil su seguimiento.

\subsection{Sprint 0 - }
\imagen{tareas_sprint_0}{Tareas sprint 0}
\imagen{burndown_sprint_0}{Burndown chart sprint 0}
\subsection{Sprint 1 - }
\imagen{tareas_sprint_1}{Tareas sprint 1}
\imagen{burndown_sprint_1}{Burndown chart sprint 1}
\subsection{Sprint 2 - }
\imagen{tareas_sprint_2}{Tareas sprint 2}
\imagen{burndown_sprint_2}{Burndown chart sprint 2}
\subsection{Sprint 3 - }
\imagen{tareas_sprint_3}{Tareas sprint 3}
\imagen{burndown_sprint_3}{Burndown chart sprint 3}
\subsection{Sprint 4 - }
\imagen{tareas_sprint_4}{Tareas sprint 4}
\imagen{burndown_sprint_4}{Burndown chart sprint 4}
\subsection{Sprint 5 - }
\imagen{tareas_sprint_5}{Tareas sprint 5}
\imagen{burndown_sprint_5}{Burndown chart sprint 5}
\subsection{Sprint 6 - }
\imagen{tareas_sprint_6}{Tareas sprint 6}
\imagen{burndown_sprint_6}{Burndown chart sprint 6}
\subsection{Sprint 7 - }
\imagen{tareas_sprint_7}{Tareas sprint 7}
\imagen{burndown_sprint_7}{Burndown chart sprint 7}

\section{Estudio de viabilidad}

\subsection{Viabilidad Técnica}

Las tecnologías utilizadas en este trabajo se emplean ampliamente en muchos proyectos de software, y 
tienen un amplio soporte por parte de su comunidad de desarrollo. Son conocidas por su eficiencia y 
flexibilidad en el desarrollo web.\\
Basándonos en los requisitos del proyecto las tecnologías elegidas para el desarrollo del trabajo 
resultan adecuadas para su correcto funcionamiento.\\
La mayor complejidad del trabajo se encuentra en el modelo LLM que se gestiona desde una instancia ajena al 
resto del proyecto de desarrollo lo que nos facilita su gestión.\\
Para el caso concreto de OpenAI resulta muy útil su uso pero hay que tener en cuenta los costes asociados 
así, como sus políticas de uso.

\subsection{Viabilidad económica}
La viabilidad económica de un proyecto de software construido con las tecnologías 
mencionadas dependerá de varios factores, incluyendo el alcance y los objetivos del proyecto, 
el mercado al que se dirige y la estrategia de monetización.

\subsubsection{Desarrollo}
En este apartado vamos a calcular el coste de desarrollo del proyecto.
Por un lado tenemos el coste del desarrollador.\\
Se han empleado alrededor de 4 meses para completar el trabajo, vamos a considerar 
un 50% por ciento de dedicación.\\
Por lo tanto tenemos unas 20 horas semanales durante 16 semanas.
El sueldo medio por hora de un desarrollador en España es de unos \href{https://es.talent.com}{15 euros} 
Así el salario total ha sido: 20*16*15 = 4800 euros
A esto hay que agregar el coste de la cotización a la seguridad social.
En estos momentos la cotización por parte de la empresa se distribuye en 
\footnote{https://www.seg-social.es/wps/portal/wss/internet/Trabajadores/CotizacionRecaudacionTrabajadores/36537}:
\begin{itemize}
    \item Contigencias: 23,6%
    \item Desempleo: 5,5%
    \item FOGASA: 0,20
    \item Formación profesional (FP): 0,60%
    \item Mecanismo de equidad intergeneracional (MEI): 0,5%
\end{itemize}
Asi el coste total será:

4800(salario) + 1132,8 (contigencias) + 264 (desempleo) + 9,6 (FOGASA) + 28,8 (FP) + 24 (MEI) = 6259,2

De la misma forma se valora el coste de asesoría (tutores), contando 2 horas a la
 semana con un coste de 40 euros la hora. 
En este caso el salario total sería:
40*2*16*2 = 2560

Y el coste total: 

2560(salario) + 604,16(contigencias) +  140 (desempleo) + 5,12 (FOGASA) + 15,36 (FP) + 12,8 (MEI) = 3337,44

\subsubsection{Software}
Todo el software utilizado en este proyecto es gratuito para aplicaciones no comerciales pero tienen un 
coste para aplicaciones comerciales excepto OpenAI que siempre tiene un coste por uso.

\begin{tabular}{r r}
    Software & Coste\\
    Python & Gratuito \\
    node.js & Gratuito\\
    Docker & Gratuito\\
    Vue & Gratuito\\
    Postgres & Gratuito\\
    OpenAI & Por uso\\
\end{tabular}

Para calcular el coste de OpenAI tenemos que tener en cuenta que éste se calcula según el 
número de \href{https://platform.openai.com/tokenizer}{tokens} enviados/recibidos.\footnote{fuente: https://openai.com/pricing}

\begin{tabular}{r r r}
    Model&Input&Output\\
    gpt-3.5-turbo-1106 & \$0.0010 / 1K tokens & \$0.0020 / 1K tokens\\
    gpt-3.5-turbo-instruct & \$0.0015 / 1K tokens & \$0.0020 / 1K tokens\\        
\end{tabular}

1000 tokens son alrededor de 750 palabras.

\subsubsection{Hardware}
Acerca del hardware utilizado en el proyecto, se ha utilizado un alojamiento llamado \href{https://contabo.com/en/}{Contabo}.\\
Este alojamiento tiene un coste de 4,5 euros al mes (+IVA = 5,45€). 
Ese precio incluye un máquina virtual linux (Ubuntu),
con 8Gb de Memoria RAM y 200Gb de disco duro.\\
Esta configuración es suficiente para el proyecto realizado.

El resto de hardware utilizado se considera ya amortizado en proyectos 
pasados y por tanto no se va a contar para este.


\subsubsection{Otros}

Otros costes asociados a este desarrollo serían:
Conexión a internet (40€ al mes)
El alquiler del dominio, que ha costado 1€ al año en \href{https://www.arsys.es/}{arsys}.
El certificado SSL se ha obtenido gratuitamente de \href{https://letsencrypt.org/}{letsencrypt}.

\subsubsection{Total}

\begin{tabular}{r r}
    Concepto & Coste\\
    Desarollo &  9596,54\\
    Software & 10€\\
    Hardware & 20€\\
    Otros & 161€\\
    Total & 9787,54\\

\end{tabular}

\subsection{Viabilidad legal}

Para estudiar la viabilidad legal del proyecto vamos a analizar las licencias de las
herramientas utilizadas.

\begin{tabular}{r r}
    Software & Licencia\\
    Python & PSF (compatible con GPL) \\
    Flask & BSD\\
    SQLAlchemy & MIT\\
    Github & GNU\\
    node.js & MIT\\
    Docker & Apache 2.0\\
    Vue & MIT\\
    Postgres & Licencia postgres\\
    OpenAI & Propietaria\\
    Visual Studio Code & MIT \\
\end{tabular}

Como vemos todas las herramientas utilizadas durante el desarrollo son Open Source, 
excepto OpenAI.