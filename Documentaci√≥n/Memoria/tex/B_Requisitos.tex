\apendice{Especificación de Requisitos}
\section{Introducción}
\section{Objetivos generales}

Crear una aplicación que permita realizar análisis de sentimiento a textos extraídos de reseñas de Amazon.
Almacenar dicha información en una base de datos para que sea fácilmente accesible.
Mostrar dicha información con gráficos y tablas para poder analizarlos.

Crear un manual de usuario. 

\section{Catálogo de requisitos}
\subsection{Requisitos funcionales}
\begin{itemize}
	\tightlist
	\item \textbf{RF1 - Registro de usuarios}: La aplicación debe permitir dar de alta usuarios y asignarles un rol.
	\item \textbf{RF2 - Login de usuarios}: La aplicación debe ser capaz de dadas las credenciales de un usuario permitirle acceder a las funcionalidades. 
	\item \textbf{RF3 - Información de la aplicación}: El usuario debe poder obtener información sobre el funcionamiento de la aplicación. 
	\item \textbf{RF4 - Analizar textos}: La aplicación debe ser capaz de analizar textos.
	\begin{itemize}
		\item \textbf{RF4.1 - Cargar reseñas de un archivo}: La aplicación debe poder cargar las reseñas de un archivo de texto
		\item \textbf{RF4.2 - Puntuar los textos}: La aplicación debe realizar el análisis del texto obtenido y devolver un valor entero entre 1 y 5 dependiendo del sentimiento del mismo, siendo 1 un sentimiento negativo y 5 un sentimiento positivo.     
		\item \textbf{RF4.3 - Almacenar resultados}: La aplicación debe almacenar las puntuaciones junto con el texto analizado, fecha en la que se escribió y un identificador. 
	\end{itemize}
	\item \textbf{RF5 - Buscar resultados por identificador}: El usuario debe poder buscar los resultados de un identificador con unas fechas seleccionadas. 
	\begin{itemize}
		\item \textbf{RF5.1 - Buscar por producto}: El usuario debe poder introducir el nombre de un producto y mostrar las reseñas del mismo. 
		\item \textbf{RF5.2 - Buscar por usuario}: El usuario debe poder buscar un usuario y mostrar la información de las reseñas que ha creado. 
		\item \textbf{RF5.3 - Buscar por texto}: El usuario debe poder buscar una palabra para ver los resultados del análisis de los textos en los que está esa palabra.
	\end{itemize}
	\item \textbf{RF6 - Matriz de confusión}: Debe mostrar la matriz de confusión para los datos seleccionados, comparando los resultados obtenidos y la calificación dada por el usuario.
	\item \textbf{RF7 - Configurar idioma}: El usuario debe poder cambiar el idioma de la aplicación. 
\end{itemize}      
\subsection{Requisitos no funcionales}
\begin{itemize}
	\item \textbf{RNF1 - Usabilidad}: La aplicación debe ser amigable para que la experiencia del usuario sea lo más positiva posible. Además debe poder adaptarse a diferentes formatos de pantalla. 
	\item \textbf{RNF2 - Eficiencia}: El almacenamiento de resultados en la base de datos y la respuesta de la aplicación al navegar entre ventanas, sobre todo, al mostrar los gráficos debe ser lo más rápido posible. 
	\item \textbf{RNF3 - Disponibilidad}: La aplicación debe estar disponible durante el mayor tiempo posible y en la mayoría de lugares. 
	\item \textbf{RNF4 - Escalabilidad}: La aplicación debe estar preparada para soportar nuevos desarrollos que generen mayor cantidad de trabajo. 
	\item \textbf{RNF5 - Confiabilidad}: La aplicación cumplirá la función para la que se ha creado. 
	\item \textbf{RNF6 - Mantenibilidad}: La aplicación debe permitir cambios y correcciones de forma eficaz.
	\item \textbf{RNF7 - Internacionalización}: La aplicación debe soportar varios idiomas.
	\item \textbf{RNF8 - Seguridad}: La aplicación debe soportar usuarios con diferentes roles que tendrán acceso a diferentes funcionalidades.
	
\end{itemize}


\section{Casos de uso}

% Caso de Uso 1 -> Registro de usuarios.
\begin{table}[p]
	\centering
	\begin{tabularx}{\linewidth}{ p{0.21\columnwidth} p{0.71\columnwidth} }
		\toprule
		\textbf{CU-1}    & \textbf{Registro de usuarios}\\
		\toprule
		\textbf{Versión}              & 1.0    \\
		\textbf{Autor}                & Teodoro Ricardo García Sánchez \\
		\textbf{Requisitos asociados} & RF-1 \\
		\textbf{Descripción}          & Un usuario con el rol de administrador podrá crear cuentas de usuario y asignarles un rol. \\
		\textbf{Precondición}         & El usuario tiene que tener el rol de administrador \\
		\textbf{Acciones}             &
		\begin{enumerate}
			\def\labelenumi{\arabic{enumi}.}
			\tightlist
			\item El usuario administrador accede a la interfaz de la aplicación.
			\item Accede a la opción Crear usuarios
			\item Introduce los datos del usuarios
			\item Confirma los cambios.
		\end{enumerate}\\
		\textbf{Postcondición}        & La cuenta de usuario no debe existir con antelación \\
		\textbf{Excepciones}          & 
		\begin{enumerate}
			\item El usuario ya existe
			\item Complejidad de la contraseña no válida
			\item Datos incorrectos
		\end{enumerate}\\
		\textbf{Importancia}          & Alta \\
		\bottomrule
	\end{tabularx}
	\caption{CU-1 Registro de usuarios.}
\end{table}
% Caso de Uso 2 -> Login de usuarios.
\begin{table}[p]
	\centering
	\begin{tabularx}{\linewidth}{ p{0.21\columnwidth} p{0.71\columnwidth} }
		\toprule
		\textbf{CU-2}    & \textbf{Login de usuarios}\\
		\toprule
		\textbf{Versión}              & 1.0    \\
		\textbf{Autor}                & Teodoro Ricardo García Sánchez \\
		\textbf{Requisitos asociados} & RF-2 \\
		\textbf{Descripción}          & El usuario accede a la aplicación tendiendo \\
										acceso a la parte que tengo acceso según su rol. \\
		\textbf{Precondición}         & Usuario registrado \\
		\textbf{Acciones}             &
		\begin{enumerate}
			\def\labelenumi{\arabic{enumi}.}
			\tightlist
			\item Accede a la pantalla de login 
			\item Introduce nombre y contraseña
			\item Pulsa en Aceptar
		\end{enumerate}\\
		\textbf{Postcondición}        & La contraseña debe coincidir \\
		\textbf{Excepciones}          & 
		\begin{enumerate}
			\def\labelenumi{\arabic{enumi}.}
			\tightlist
			\item El usuario no existe 
			\item La contraseña no es válida 
		\end{enumerate}\\
		\textbf{Importancia}          & Alta\\
		\bottomrule
	\end{tabularx}
	\caption{CU-2 Login de usuarios.}
\end{table}
% Caso de Uso 3 -> Información de la aplicación.
\begin{table}[p]
	\centering
	\begin{tabularx}{\linewidth}{ p{0.21\columnwidth} p{0.71\columnwidth} }
		\toprule
		\textbf{CU-3}    & \textbf{Información de la aplicación}\\
		\toprule
		\textbf{Versión}              & 1.0    \\
		\textbf{Autor}                & Teodoro Ricardo García Sánchez \\
		\textbf{Requisitos asociados} & RF-3 \\
		\textbf{Descripción}          & Permite al usuario obtener información sobre la configuración de la aplicación \\
		\textbf{Precondición}         & Usuario logado \\
		\textbf{Acciones}             &
		\begin{enumerate}
			\def\labelenumi{\arabic{enumi}.}
			\tightlist
			\item El usuario accede a la opción información.
		\end{enumerate}\\
		\textbf{Postcondición}        &  \\
		\textbf{Excepciones}          &  \\
		\textbf{Importancia}          & Baja \\
		\bottomrule
	\end{tabularx}
	\caption{CU-3 Información de la aplicación.}
\end{table}
% Caso de Uso 4 -> Cargar reseñas de un archivo.
\begin{table}[p]
	\centering
	\begin{tabularx}{\linewidth}{ p{0.21\columnwidth} p{0.71\columnwidth} }
		\toprule
		\textbf{CU-4}    & \textbf{Cargar reseñas de un archivo}\\
		\toprule
		\textbf{Versión}              & 1.0    \\
		\textbf{Autor}                & Teodoro Ricardo García Sánchez \\
		\textbf{Requisitos asociados} & RF4.1 \\
		\textbf{Descripción}          & El usuario puede cargar reseñas de un archivo \\
		\textbf{Precondición}         & Usuario logado \\
		\textbf{Acciones}             &
		\begin{enumerate}
			\def\labelenumi{\arabic{enumi}.}
			\tightlist
			\item Pasos del CU
			\item Pasos del CU (añadir tantos como sean necesarios)
		\end{enumerate}\\
		\textbf{Postcondición}        & Postcondiciones (podría haber más de una) \\
		\textbf{Excepciones}          & Excepciones \\
		\textbf{Importancia}          & Alta o Media o Baja... \\
		\bottomrule
	\end{tabularx}
	\caption{CU-04 Cargar reseñas de un archivo.}
\end{table}
% Caso de Uso 5 -> Puntuar los textos.
\begin{table}[p]
	\centering
	\begin{tabularx}{\linewidth}{ p{0.21\columnwidth} p{0.71\columnwidth} }
		\toprule
		\textbf{CU-5}    & \textbf{Puntuar los textos}\\
		\toprule
		\textbf{Versión}              & 1.0    \\
		\textbf{Autor}                & Teodoro Ricardo García Sánchez \\
		\textbf{Requisitos asociados} & RF4.2 \\
		\textbf{Descripción}          & La aplicación debe realizar el análisis del texto obtenido
										 y devolver un valor entero entre 1 y 5 dependiendo del sentimiento del mismo, siendo 1 un sentimiento negativo y 5 un sentimiento positivo. \\
		\textbf{Precondición}         &  \\
		\textbf{Acciones}             &
		\begin{enumerate}
			\def\labelenumi{\arabic{enumi}.}
			\tightlist
			\item Pasos del CU
			\item Pasos del CU (añadir tantos como sean necesarios)
		\end{enumerate}\\
		\textbf{Postcondición}        & Postcondiciones (podría haber más de una) \\
		\textbf{Excepciones}          & Excepciones \\
		\textbf{Importancia}          & Alta o Media o Baja... \\
		\bottomrule
	\end{tabularx}
	\caption{CU-5 Puntuar los textos.}
\end{table}
% Caso de Uso 6 -> Almacenar resultados.
\begin{table}[p]
	\centering
	\begin{tabularx}{\linewidth}{ p{0.21\columnwidth} p{0.71\columnwidth} }
		\toprule
		\textbf{CU-6}    & \textbf{Alamacenar resultados}\\
		\toprule
		\textbf{Versión}              & 1.0    \\
		\textbf{Autor}                & Teodoro Ricardo García Sánchez \\
		\textbf{Requisitos asociados} & RF4.3 \\
		\textbf{Descripción}          & La aplicación debe almacenar las puntuaciones junto con el texto analizado, fecha en la que se escribió y un identificador. \\
		\textbf{Precondición}         & Precondiciones (podría haber más de una) \\
		\textbf{Acciones}             &
		\begin{enumerate}
			\def\labelenumi{\arabic{enumi}.}
			\tightlist
			\item Pasos del CU
			\item Pasos del CU (añadir tantos como sean necesarios)
		\end{enumerate}\\
		\textbf{Postcondición}        & Postcondiciones (podría haber más de una) \\
		\textbf{Excepciones}          & Excepciones \\
		\textbf{Importancia}          & Alta o Media o Baja... \\
		\bottomrule
	\end{tabularx}
	\caption{CU-6 Almacenar resultados.}
\end{table}
% Caso de Uso 7 -> Buscar resultados por producto.
\begin{table}[p]
	\centering
	\begin{tabularx}{\linewidth}{ p{0.21\columnwidth} p{0.71\columnwidth} }
		\toprule
		\textbf{CU-7}    & \textbf{Buscar resultado por producto}\\
		\toprule
		\textbf{Versión}              & 1.0    \\
		\textbf{Autor}                & Teodoro Ricardo García Sánchez \\
		\textbf{Requisitos asociados} & RF5.1 \\
		\textbf{Descripción}          & El usuario debe poder introducir el nombre de un producto y mostrar las reseñas del mismo \\
		\textbf{Precondición}         & Precondiciones (podría haber más de una) \\
		\textbf{Acciones}             &
		\begin{enumerate}
			\def\labelenumi{\arabic{enumi}.}
			\tightlist
			\item Pasos del CU
			\item Pasos del CU (añadir tantos como sean necesarios)
		\end{enumerate}\\
		\textbf{Postcondición}        & Postcondiciones (podría haber más de una) \\
		\textbf{Excepciones}          & Excepciones \\
		\textbf{Importancia}          & Alta o Media o Baja... \\
		\bottomrule
	\end{tabularx}
	\caption{CU-7 Buscar resultados por producto.}
\end{table}
% Caso de Uso 8 -> Buscar por nombre de usuario.
\begin{table}[p]
	\centering
	\begin{tabularx}{\linewidth}{ p{0.21\columnwidth} p{0.71\columnwidth} }
		\toprule
		\textbf{CU-8}    & \textbf{Buscar por nombre de usuario}\\
		\toprule
		\textbf{Versión}              & 1.0    \\
		\textbf{Autor}                & Teodoro Ricardo García Sánchez \\
		\textbf{Requisitos asociados} & RF5.2 \\
		\textbf{Descripción}          & Se debe poder buscar un usuario y mostrar la información de las reseñas que ha creado. \\
		\textbf{Precondición}         & Precondiciones (podría haber más de una) \\
		\textbf{Acciones}             &
		\begin{enumerate}
			\def\labelenumi{\arabic{enumi}.}
			\tightlist
			\item Pasos del CU
			\item Pasos del CU (añadir tantos como sean necesarios)
		\end{enumerate}\\
		\textbf{Postcondición}        & Postcondiciones (podría haber más de una) \\
		\textbf{Excepciones}          & Excepciones \\
		\textbf{Importancia}          & Alta o Media o Baja... \\
		\bottomrule
	\end{tabularx}
	\caption{CU-8 Buscar nombre de usuario.}
\end{table}
% Caso de Uso 9 -> Buscar por texto.
\begin{table}[p]
	\centering
	\begin{tabularx}{\linewidth}{ p{0.21\columnwidth} p{0.71\columnwidth} }
		\toprule
		\textbf{CU-9}    & \textbf{Buscar por texto}\\
		\toprule
		\textbf{Versión}              & 1.0    \\
		\textbf{Autor}                & Teodoro Ricardo García Sánchez \\
		\textbf{Requisitos asociados} & RF5.3 \\
		\textbf{Descripción}          & El usuario debe poder buscar una palabra para ver los resultados del análisis de los textos en los que está esa palabra. \\
		\textbf{Precondición}         & Precondiciones (podría haber más de una) \\
		\textbf{Acciones}             &
		\begin{enumerate}
			\def\labelenumi{\arabic{enumi}.}
			\tightlist
			\item Pasos del CU
			\item Pasos del CU (añadir tantos como sean necesarios)
		\end{enumerate}\\
		\textbf{Postcondición}        & Postcondiciones (podría haber más de una) \\
		\textbf{Excepciones}          & Excepciones \\
		\textbf{Importancia}          & Alta o Media o Baja... \\
		\bottomrule
	\end{tabularx}
	\caption{CU-9 Buscar por texto.}
\end{table}
% Caso de Uso 10 -> Matriz de confusión.
\begin{table}[p]
	\centering
	\begin{tabularx}{\linewidth}{ p{0.21\columnwidth} p{0.71\columnwidth} }
		\toprule
		\textbf{CU-8}    & \textbf{Matriz de confusión}\\
		\toprule
		\textbf{Versión}              & 1.0    \\
		\textbf{Autor}                & Teodoro Ricardo García Sánchez \\
		\textbf{Requisitos asociados} & RF-xx, RF-xx \\
		\textbf{Descripción}          & La descripción del CU \\
		\textbf{Precondición}         & Precondiciones (podría haber más de una) \\
		\textbf{Acciones}             &
		\begin{enumerate}
			\def\labelenumi{\arabic{enumi}.}
			\tightlist
			\item Pasos del CU
			\item Pasos del CU (añadir tantos como sean necesarios)
		\end{enumerate}\\
		\textbf{Postcondición}        & Postcondiciones (podría haber más de una) \\
		\textbf{Excepciones}          & Excepciones \\
		\textbf{Importancia}          & Alta o Media o Baja... \\
		\bottomrule
	\end{tabularx}
	\caption{CU-1 Nombre del caso de uso.}
\end{table}

% Caso de Uso 11 -> Consultar Experimentos.
\begin{table}[p]
	\centering
	\begin{tabularx}{\linewidth}{ p{0.21\columnwidth} p{0.71\columnwidth} }
		\toprule
		\textbf{CU-1}    & \textbf{Ejemplo de caso de uso}\\
		\toprule
		\textbf{Versión}              & 1.0    \\
		\textbf{Autor}                & Teodoro Ricardo García Sánchez \\
		\textbf{Requisitos asociados} & RF-xx, RF-xx \\
		\textbf{Descripción}          & La descripción del CU \\
		\textbf{Precondición}         & Precondiciones (podría haber más de una) \\
		\textbf{Acciones}             &
		\begin{enumerate}
			\def\labelenumi{\arabic{enumi}.}
			\tightlist
			\item Pasos del CU
			\item Pasos del CU (añadir tantos como sean necesarios)
		\end{enumerate}\\
		\textbf{Postcondición}        & Postcondiciones (podría haber más de una) \\
		\textbf{Excepciones}          & Excepciones \\
		\textbf{Importancia}          & Alta o Media o Baja... \\
		\bottomrule
	\end{tabularx}
	\caption{CU-1 Nombre del caso de uso.}
\end{table}



