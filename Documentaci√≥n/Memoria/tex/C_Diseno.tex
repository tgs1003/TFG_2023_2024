\apendice{Especificación de diseño}

\section{Introducción}

\section{Diseño de datos}
Para almacenar la información necesaria para la aplicación he diseñado una base de datos 
que sirva para facilitar el procesado de las reseñas y su posterior análisis 
por parte de los usuarios de la aplicación:

\imagen{Arquitectura Base de datos}{Arquitectura de la base de datos}

Como vemos almacenamos la información relativa a los conjuntos de datos en la tabla "datasets".
En esta tabla almacenamos la información necesaria para cargarlo como es el tipo (type) y el payload.
Según el tipo de conjunto de datos el método para cargarlo será diferente.
Una vez iniciado el proceso de carga, la información relativa a las reseñas se almacenan en la tabla "Review", "Products" y "ReviewUsers".

Por último, se almacena en la tabla "sentiments" la información relativa al análisis de sentimiento.
Se incluye la información obtenida del modelo así como el tiempo que ha tardado y el número de tokens procesados.
Este último campo (tokens) se puede utilizar para calcular el coste en el caso de OpenAI.
También se indica si la respuesta del modelo se considera errónea por tener un formato incorrecto. 
En este caso la respuesta del modelo se almacena en el campo "source".

\section{Diseño procedimental}

\imagen{Diagrama de secuencia login}{Login}
\imagen{Diagrama de secuencia admin}{Administración}
\imagen{Diagrama de secuencia gestor}{Gestor}


\section{Diseño arquitectónico}

\subsection{Arquitectura de la aplicación}

En el caso de la arquitectura de la aplicación, vemos que está centrada en la base de datos.
Desde el módulo de administración se pueden gestionar los usuarios y las reseñas.
Desde el módulo de servicios se cargan las reseñas y se gestionan los diferentes procesos de análisis.
A través de un servicio web se expone al usuario los diferentes informes de resultados.

\imagen{Arquitectura de la aplicación}{Arquitectura de la aplicación}

\subsection{Interfaz de usuario}
En la pantalla de login debe aparecer el logo de la aplicación 
junto con la información necesaria para poder acceder a la aplicación.
Se puede mostrar también una pequeña introdución sobre el proposito de la misma y 
un pequeño vínculo para acceder al registro como usuario.
\imagen{login}{Pantalla de login.}
En esta pantalla se mostrará la información relativa a las reseñas y 
al análisis de sentimiento realizado hasta el momento.
\imagen{Inicio}{Pantalla usuario gestor.}
En la pantalla de administración se podrá gestionar los usuarios y las reseñas.
\imagen{administración}{Pantalla de administración.}
