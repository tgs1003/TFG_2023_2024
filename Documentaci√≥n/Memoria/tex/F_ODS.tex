\apendice{Anexo de sostenibilización curricular}

\section{Introducción}
Durante la realización de este trabajo he tenido presente la importancia 
que tiene que un desarrollo sea sostenible.

Para hacer la aplicación he seguido una serie de técnicas reconocidas de forma 
general como buenas prácticas y que implican un enfoque orientado a la sostenibilidad.
Entre otras he seguido las siguientes pautas:
\begin{itemize}
    \item Uso herramientas de código abierto: Al usar herramientas de código abierto no solo 
          ahorramos el coste de las licencias sino que contribuimos a su desarrollo como herramientas universales. 
    \item Uso de patrones de desarrollo: los patrones de desarrollo son soluciones simples 
          a problemas complejos. Haciendo uso de éstas conseguimos reducir 
          la complejidad del desarrollo y lo hace más sostenible y 
          accesible al resto de la comunidad de desarrolladores. 
    \item Código autocomentado: código fácil de leer y entender para facilitar el 
            apredidaje de futuros desarrolladores y su mantenimiento.
    \item Código libre y disponible en Github: haciendo el código abierto a todo el mundo facilitamos 
            el desarrollo de futuros proyectos reduciendo su coste y complejidad (en muchos casos). 
    \item Uso de marcos de desarrollo de uso generalizado: usar marcos de desarrollo de uso común hace 
            que nuestra aplicación sea más mantenible y extensible para otros desarrolladores.
    \item Gestión de proyectos ágil: al seguir una metodología ágil se consigue reducir la complejidad del 
        proyecto y adaptarse mejor a los cambios. Si en medio de un sprint surgía algún impedimento 
        podía cambiar su planificación sin tener efecto en la velocidad del equipo. 
    \item Principios KISS: los principos KISS se basan en mantener los desarrollos simples y 
        no hacer desarrollos que no sean necesarios.
    \item Enfoque Devops: al hacer el despliegue de la aplicación por configuración y hacerse desde el 
        primer momento éste presenta menos problemas al final del proyecto y la aplicación está publicada 
        desde el primer momento. Esto también está en consonancia con los principios de la gestión ágil. 
    \item Escalabilidad: al usar docker y ngix la aplicación está preparada para la escalabilidad desde el primer momento.
    \item Interfaz en múltiples idiomas: el tener una interfaz preparada para varios idiomas hace que sea más 
        accesible a un grupo más amplio de usuarios.
\end{itemize}

Desde el punto de vista de costes de producción, debido a los bajos 
requisitos que tiene para poder funcionar en un entorno de producción, 
se puede alojar en multitud de servidores en función de su utilización.

Al estar configurada para funcionar en docker puede funcionar en multitud de entornos y configuraciones.
Es fácil de escalar y se puede configurar en modo pago por uso (en muchos alojamientos).
De esta forma el retorno de la inversión será muy rápido.

Como hemos visto en la sección de Viabilidad económica el único gasto fijo es el alojamiento (5€ al mes).
Los gastos variables son el uso de OpenAI que depende exclusivamente del uso.

Desde un punto de vista social tener una herramienta que ayude a las empresas a identificar los 
puntos de mejora de un producto o servicio es bueno para todos. 
Por un lado los la empresa mejora sus resultados al poder identificar los puntos de mejora demandados.
Y por otro el cliente se siente escuchado y recibe un mejor servicio.

