\apendice{Anexo de sostenibilización curricular}

\section{Introducción}
Este anexo incluirá una reflexión personal del alumnado sobre los aspectos de la sostenibilidad que se abordan en el trabajo.
Se pueden incluir tantas subsecciones como sean necesarias con la intención de explicar las competencias de sostenibilidad adquiridas durante el alumnado y aplicadas al Trabajo de Fin de Grado.

Más información en el documento de la CRUE \url{https://www.crue.org/wp-content/uploads/2020/02/Directrices_Sosteniblidad_Crue2012.pdf}.
Este anexo tendrá una extensión comprendida entre 600 y 800 palabras.

La sostenibilidad de un proyecto de software como el mencionado dependerá de varios factores, y es crucial considerar tanto los aspectos económicos como los sociales y ambientales. Aquí hay algunas reflexiones sobre la sostenibilidad del proyecto:

Económica:
Modelo de Negocio Sostenible: Un modelo de negocio sólido y sostenible es esencial. 
Evalúa la capacidad del proyecto para generar ingresos de manera constante 
y asegúrate de que el modelo de monetización elegido sea viable a largo plazo.
Gestión de Costos: Controlar eficientemente los costos de desarrollo, 
mantenimiento y operación es clave para la sostenibilidad económica. 
Considera la posibilidad de utilizar servicios en la nube de manera eficiente 
y evalúa constantemente la rentabilidad de las herramientas y tecnologías utilizadas.
Social:
Impacto Positivo: Evalúa cómo el proyecto puede tener un impacto positivo en la sociedad. 
Por ejemplo, si el software ayuda a las empresas a comprender y mejorar la satisfacción del cliente, 
esto podría contribuir a un impacto social positivo.
Accesibilidad: Asegúrate de que el software sea accesible para la mayor cantidad de personas posible. 
Considera aspectos como la usabilidad, la inclusión de personas con discapacidades 
y la disponibilidad en diferentes idiomas.
Ética en la Inteligencia Artificial: 
Si el proyecto utiliza inteligencia artificial (como OpenAI), 
es crucial considerar aspectos éticos. 
Asegúrate de que el uso de IA sea transparente, justo y ético, y evita sesgos o discriminación.
Ambiental:
Eficiencia Energética: Evalúa la eficiencia energética del software y las tecnologías utilizadas. 
La sostenibilidad ambiental implica minimizar el consumo de recursos, incluida la energía.
Impacto del Desarrollo: Considera el impacto ambiental durante el desarrollo del software. 
Esto podría incluir prácticas de desarrollo sostenibles, 
como la reducción de residuos de código o la eficiencia en el uso de recursos de desarrollo.
Comunidad y Colaboración:
Involucramiento de la Comunidad: Fomenta la participación y retroalimentación de la comunidad. 
Una comunidad activa puede contribuir al éxito continuo del proyecto 
y proporcionar ideas valiosas para mejoras sostenibles.
Colaboración Abierta: Considera la posibilidad de colaborar con otros 
proyectos de código abierto y contribuir al ecosistema más amplio. 
La colaboración puede generar sinergias y mejorar la sostenibilidad a largo plazo.
En resumen, para lograr la sostenibilidad de un proyecto de software, 
es necesario equilibrar aspectos económicos, sociales y ambientales. 
La transparencia, la ética y la consideración de los impactos a largo plazo 
son esenciales para construir y mantener un proyecto que beneficie tanto 
a los usuarios como al entorno en el que se desarrolla.